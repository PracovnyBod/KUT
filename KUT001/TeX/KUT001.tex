\documentclass[a4paper, 10pt, ]{article}

\usepackage[slovak]{babel}

% ------------------------------

\usepackage[utf8]{inputenc}
\usepackage[T1]{fontenc}

\usepackage[left=4cm,
            right=4cm,
            top=2.1cm,
            bottom=2.6cm,
            footskip=7.5mm,
            twoside,
            marginparwidth=3.0cm,
            %showframe,
            ]{geometry}

\usepackage{graphicx}
\usepackage[dvipsnames]{xcolor}
% https://en.wikibooks.org/wiki/LaTeX/Colors

% ------------------------------

\usepackage{lmodern}

\usepackage[tt={oldstyle=false,proportional=true,monowidth}]{cfr-lm}
% https://mirror.szerverem.hu/ctan/fonts/cfr-lm/doc/cfr-lm.pdf

% ------------------------------

\usepackage{amsmath}
\usepackage{amssymb}
\usepackage{amsthm}

\usepackage{booktabs}
\usepackage{multirow}
\usepackage{array}
\usepackage{dcolumn}

\usepackage{natbib}

% ------------------------------

\hyphenpenalty=6000
\tolerance=1000

\def\naT{\mathsf{T}}

% ------------------------------

\makeatletter

    \def\@seccntformat#1{\protect\makebox[0pt][r]{\csname the#1\endcsname\hspace{4mm}}}

    \def\cleardoublepage{\clearpage\if@twoside \ifodd\c@page\else
    \hbox{}
    \vspace*{\fill}
    \begin{center}
    \phantom{}
    \end{center}
    \vspace{\fill}
    \thispagestyle{empty}
    \newpage
    \if@twocolumn\hbox{}\newpage\fi\fi\fi}

    \newcommand\figcaption{\def\@captype{figure}\caption}
    \newcommand\tabcaption{\def\@captype{table}\caption}

\makeatother

% ------------------------------

\usepackage{fancyhdr}
\fancypagestyle{plain}{%
\fancyhf{} % clear all header and footer fields
% \fancyfoot[C]{\sffamily {\bfseries \thepage}\ | {\scriptsize\oznacenieCasti}}
\fancyfoot[C]{\sffamily {\bfseries \thepage}{\color{Gray}\scriptsize$\,$z$\,$\pageref{LastPage}}\ | \includegraphics[height=5pt]{./COMMONFILES/KUT_logo_v0.1.pdf}{\scriptsize\KUTporadoveCislo}}
\renewcommand{\headrulewidth}{0pt}
\renewcommand{\footrulewidth}{0pt}}
\pagestyle{plain}

% ------------------------------

\usepackage{titlesec}
\titleformat{\paragraph}[hang]{\sffamily  \bfseries}{}{0pt}{}
\titlespacing*{\paragraph}{0mm}{3mm}{1mm}
\titlespacing*{\subparagraph}{0mm}{3mm}{1mm}

\titleformat*{\section}{\sffamily\Large\bfseries}
\titleformat*{\subsection}{\sffamily\large\bfseries}
\titleformat*{\subsubsection}{\sffamily\normalsize\bfseries}


% ------------------------------

\PassOptionsToPackage{hyphens}{url}
\usepackage[pdfauthor={},
            pdftitle={},
            pdfsubject={},
            pdfkeywords={},
            % hidelinks,
            colorlinks=false,
            breaklinks,
            ]{hyperref}


% ------------------------------

\graphicspath{%
{../fig_standalone/}%
{../../PY/fig/}%
{../../ML/fig/}%
{./fig/}%
}

% ------------------------------

\usepackage{enumitem}

\usepackage{lettrine}

% ------------------------------

\usepackage{lastpage}

\usepackage{microtype}

% ------------------------------

\usepackage{algorithm}
\usepackage[noend]{algpseudocode}
\makeatletter
\renewcommand{\ALG@name}{Algoritmus}
\makeatother
\usepackage{amsmath}
\usepackage{bbold}
\usepackage{calc}
\usepackage{dsfont}
\usepackage{mathtools}
\usepackage{tabto}


\newcommand{\mr}[1]{\mathrm{#1}}
\newcommand{\bs}[1]{\boldsymbol{#1}}
\newcommand{\bm}[1]{\mathbf{#1}}

\newcommand{\diff}[2]{\frac{\Delta #1}{\Delta #2}}
\newcommand{\der}[2]{\frac{d #1}{d #2}}
\newcommand{\parder}[2]{\frac{\partial #1}{\partial #2}}

\newcommand{\argmax}[0]{\mr{argmax}}
\newcommand{\diag}[0]{\mr{diag}}
\newcommand{\rank}[0]{\mr{rank}}
\newcommand{\trace}[0]{\mr{tr}}

\renewcommand{\Re}{\mr{Re}}
\renewcommand{\Im}{\mr{Im}}


\theoremstyle{definition}
\newtheorem{definition}{Definícia}[section]
\newtheorem{theorem}{Veta}[section]
\newtheorem{lemma}[theorem]{Lemma}
\newtheorem{example}{Príklad}[section]
\renewcommand*{\proofname}{Dôkaz}

% ------------------------------


% -----------------------------------------------------------------------------

\def\oznacenieCelku{Kolekcia učebných textov}

% -----------------------------------------------------------------------------

\def\KUTporadoveCislo{001}

% \def\oznacenieVerzie{v1.0}           % moznost nastavit verziu priamo tu
\def\oznacenieVerzie{\phantom{v1.0}}   % verzia sa nezobrazuje

\def\mesiacRok{február 2024}

\def\authorslabel{}






% -----------------------------------------------------------------------------

\begin{document}


% -----------------------------------------------------------------------------
% Uvodny nadpis

\noindent
\parbox[t][18mm][c]{0.3\textwidth}{%
\raisebox{-0.9\height}{%
\phantom{.}\includegraphics[height=18mm]{./COMMONFILES/URKFEIlogo.pdf}%
}%
}%
\parbox[t][18mm][c]{0.7\textwidth}{%
\raggedleft



\sffamily
\fontsize{16pt}{18pt}
\fontseries{sbc}
\selectfont

\noindent
\textcolor[rgb]{0.75, 0.75, 0.75}{\textls[25]{\oznacenieCelku}}
}%

\noindent
\parbox[t][16mm][b]{0.5\textwidth}{%
\raggedright

\color{Gray}
\sffamily


\fontsize{8pt}{10pt}
\selectfont

\authorslabel

\fontsize{6pt}{10pt}
\selectfont

urk.fei.stuba.sk

\fontsize{12pt}{12pt}
\selectfont
\mesiacRok


}%
\parbox[t][16mm][b]{0.5\textwidth}{%
\raggedleft

\sffamily

\fontsize{6pt}{6pt}
\selectfont

\textcolor[rgb]{0.68, 0.68, 0.68}{\oznacenieVerzie}


\fontsize{14pt}{14pt}
\selectfont

\bfseries

\includegraphics[height=12pt]{./COMMONFILES/KUT_logo_v0.1.pdf}%
{%
\textls[-50]{\KUTporadoveCislo}
}%

}%

% -----------------------------------------------------------------------------




\vspace{6mm}

% ---------------------------------------------
\sffamily
\bfseries
\fontsize{18pt}{21pt}
\selectfont

\begin{flushleft}
	O rozklade na diferenciálne \\rovnice prvého rádu
\end{flushleft}

\bigskip

% -----------------------------------------------------------------------------
\normalsize
\normalfont
% -----------------------------------------------------------------------------










\noindent
\lettrine[lines=1, nindent=1pt, loversize=0.0]{P}{re} 
diferenciálne rovnice, aké tu máme na mysli, platí, že každú diferenciálnu rovnicu vyššieho rádu je možné rozložiť na sústavu rovníc nižšieho rádu.



\paragraph{Diferenciálna rovnica}

Majme obyčajnú diferenciálnu rovnicu, ktorú je možné využiť na opis dynamického systému. Vo všeobecnosti:
\begin{equation}
    a_n \frac{\text{d}^n y(t)}{\text{d}t^n} 
    + a_{n-1} \frac{\text{d}^{n-1} y(t)}{\text{d}t^{n-1}}
    +
    \cdots
    +
    a_0 y(t)
    =
    b_m \frac{\text{d}^m u(t)}{\text{d}t^m} 
    + b_{m-1} \frac{\text{d}^{m-1} u(t)}{\text{d}t^{m-1}}
    +
    \cdots
    +
    b_0 u(t)
\end{equation}
kde $y(t)$ je výstupná veličina (výstupný signál) a ide teda prirodzene o časovú závislosť, funkciu v čase. Čas je označený písmenom $t$. Vstupná veličina (vstupný signál) je $u(t)$. Koeficienty $a_n, \ldots, a_0$ a $b_m, \ldots, b_0$ sú konštantné, nemenia sa v čase.

Číslo $n$ udáva najvyššiu časovú deriváciu výstupného signálu a číslo $m$ udáva najvyššiu časovú deriváciu vstupného signálu. Uvažujeme samozrejme kauzálny dynamický systém a preto $n>m$.

Poznámka: časovú deriváciu je často praktické označovať bodkou nad signálom namiesto klasického označenia $\frac{\text{d}}{\text{d}t}$. Napríklad $\frac{\text{d}^2 y(t)}{\text{d}t^2}$ označíme ako $\ddot y(t)$.

Hovoríme, že diferenciálna rovnica je $n$-tého rádu. Podľa najvyššej derivácie výstupnej veličiny. Najnižší možný rád diferenciálnej rovnice je prvý rád, $n=1$.



\paragraph{Postup rozkladu formou príkladu}

Vo všeobecnosti platí, že každú diferenciálnu rovnicu vyššieho rádu je možné rozložiť (prepísať, transformovať) na sústavu rovníc prvého rádu. Ich počet je minimálne $n$.

Ako príklad uvažujme diferenciálnu rovnicu v tvare
\begin{equation} \label{povonadr2}
    a_2 \ddot y(t) + a_1 \dot y(t) + a_0 y(t) = b_0 u(t)
\end{equation}

Úlohou je rozložiť túto diferenciálnu rovnicu druhého rádu na dve diferenciálne rovnice prvého rádu.

Pre tieto dve nové rovnice je potrebné uvažovať dve veličiny, ktoré budú na mieste neznámej v nových diferenciálnych rovniciach. Označme ich $x_1(t)$ a $x_2(t)$.

Hľadáme teda dve diferenciálne rovnice, inak vyjadrené hľadáme
\begin{align*}
    \dot x_1(t) &= \quad ? \\
    \dot x_2(t) &= \quad ? 
\end{align*}
pritom na pravej strane je potrebné mať len také členy, ktoré obsahujú len nové veličiny $x_1(t)$ a $x_2(t)$ a neobsahujú pôvodnú veličinu $y(t)$. Mimochodom, veličina $u(t)$, teda vstupný signál systému, je z hľadiska riešenia diferenciálnej rovnice známa. Neznámou je výstupná veličina $y(t)$. Signál $u(t)$ preto môže nezmenený figurovať v nových hľadaných diferenciálnych rovniciach. Dosiahnuť uvedené je možné nasledujúcim postupom. 

Ako prvé \emph{zvoľme}
\begin{equation} \label{volba1}
    x_1(t) = y(t)
\end{equation}
To znamená
\begin{equation}
    \dot x_1(t) = \dot y(t)
\end{equation}
čo však nie je v tvare aký hľadáme. Na pravej strane vystupuje pôvodná veličina $y(t)$.

Druhou voľbou preto nech je
\begin{equation} \label{volba2}
    x_2(t) = \dot y(t)
\end{equation}
pretože potom môžeme písať prvú diferenciálnu rovnicu v tvare
\begin{equation}
    \dot x_1(t) = x_2(t)
\end{equation}
Ostáva zostaviť druhú diferenciálnu rovnicu. 

Keďže sme zvolili \eqref{volba2}, tak je zrejmé, že platí
\begin{equation} 
    \dot x_2(t) = \ddot y(t)
\end{equation}
Otázkou je $\ddot y(t) = \ ?$ Odpoveďou je pôvodná diferenciálna rovnica druhého rádu. Upravme \eqref{povonadr2} na tvar
\begin{align}
     \ddot y(t) + \frac{a_1}{a_2} \dot y(t) + \frac{a_0}{a_2} y(t) &= \frac{b_0}{a_2} u(t) \\
     \ddot y(t) &= - \frac{a_1}{a_2} \dot y(t) - \frac{a_0}{a_2} y(t) +  \frac{b_0}{a_2} u(t) 
\end{align}
To znamená, že
\begin{equation}  \label{druhadr_1}
    \dot x_2(t) = - \frac{a_1}{a_2} \dot y(t) - \frac{a_0}{a_2} y(t) +  \frac{b_0}{a_2} u(t) 
\end{equation}
čo však stále nie je požadovaný tvar druhej hľadanej diferenciálnej rovnice. Na pravej strane rovnice \eqref{druhadr_1} môžu figurovať len nové veličiny $x_1(t)$ a $x_2(t)$, nie pôvodná veličina $y(t)$. Stačí si však všimnúť skôr zvolené \eqref{volba1} a \eqref{volba2}. Potom môžeme písať
\begin{equation}  \label{druhadr_2}
    \dot x_2(t) = - \frac{a_1}{a_2}  x_2(t) - \frac{a_0}{a_2} x_1(t) +  \frac{b_0}{a_2} u(t) 
\end{equation}
čo je druhá hľadaná diferenciálna rovnica prvého rádu.





\paragraph{Záver}

Diferenciálnu rovnicu druhého rádu
\begin{equation} 
    a_2 \ddot y(t) + a_1 \dot y(t) + a_0 y(t) = b_0 u(t)
\end{equation}
sme transformovali na sústavu diferenciálnych rovníc prvého rádu
\begin{align}
    \dot x_1(t) &= x_2(t) \\
    \dot x_2(t) &= - \frac{a_1}{a_2}  x_2(t) - \frac{a_0}{a_2} x_1(t) +  \frac{b_0}{a_2} u(t) 
\end{align}












% -----------------------------------------------------------------------------
\end{document}
% -----------------------------------------------------------------------------