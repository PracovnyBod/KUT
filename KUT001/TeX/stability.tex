\documentclass[a4paper, 10pt, ]{article}

\usepackage[slovak]{babel}

% ------------------------------

\usepackage[utf8]{inputenc}
\usepackage[T1]{fontenc}

\usepackage[left=4cm,
            right=4cm,
            top=2.1cm,
            bottom=2.6cm,
            footskip=7.5mm,
            twoside,
            marginparwidth=3.0cm,
            %showframe,
            ]{geometry}

\usepackage{graphicx}
\usepackage[dvipsnames]{xcolor}
% https://en.wikibooks.org/wiki/LaTeX/Colors

% ------------------------------

\usepackage{lmodern}

\usepackage[tt={oldstyle=false,proportional=true,monowidth}]{cfr-lm}
% https://mirror.szerverem.hu/ctan/fonts/cfr-lm/doc/cfr-lm.pdf

% ------------------------------

\usepackage{amsmath}
\usepackage{amssymb}
\usepackage{amsthm}

\usepackage{booktabs}
\usepackage{multirow}
\usepackage{array}
\usepackage{dcolumn}

\usepackage{natbib}

% ------------------------------

\hyphenpenalty=6000
\tolerance=1000

\def\naT{\mathsf{T}}

% ------------------------------

\makeatletter

    \def\@seccntformat#1{\protect\makebox[0pt][r]{\csname the#1\endcsname\hspace{4mm}}}

    \def\cleardoublepage{\clearpage\if@twoside \ifodd\c@page\else
    \hbox{}
    \vspace*{\fill}
    \begin{center}
    \phantom{}
    \end{center}
    \vspace{\fill}
    \thispagestyle{empty}
    \newpage
    \if@twocolumn\hbox{}\newpage\fi\fi\fi}

    \newcommand\figcaption{\def\@captype{figure}\caption}
    \newcommand\tabcaption{\def\@captype{table}\caption}

\makeatother

% ------------------------------

\usepackage{fancyhdr}
\fancypagestyle{plain}{%
\fancyhf{} % clear all header and footer fields
% \fancyfoot[C]{\sffamily {\bfseries \thepage}\ | {\scriptsize\oznacenieCasti}}
\fancyfoot[C]{\sffamily {\bfseries \thepage}{\color{Gray}\scriptsize$\,$z$\,$\pageref{LastPage}}\ | \includegraphics[height=5pt]{./COMMONFILES/KUT_logo_v0.1.pdf}{\scriptsize\KUTporadoveCislo}}
\renewcommand{\headrulewidth}{0pt}
\renewcommand{\footrulewidth}{0pt}}
\pagestyle{plain}

% ------------------------------

\usepackage{titlesec}
\titleformat{\paragraph}[hang]{\sffamily  \bfseries}{}{0pt}{}
\titlespacing*{\paragraph}{0mm}{3mm}{1mm}
\titlespacing*{\subparagraph}{0mm}{3mm}{1mm}

\titleformat*{\section}{\sffamily\Large\bfseries}
\titleformat*{\subsection}{\sffamily\large\bfseries}
\titleformat*{\subsubsection}{\sffamily\normalsize\bfseries}


% ------------------------------

\PassOptionsToPackage{hyphens}{url}
\usepackage[pdfauthor={},
            pdftitle={},
            pdfsubject={},
            pdfkeywords={},
            % hidelinks,
            colorlinks=false,
            breaklinks,
            ]{hyperref}


% ------------------------------

\graphicspath{%
{../fig_standalone/}%
{../../PY/fig/}%
{../../ML/fig/}%
{./fig/}%
}

% ------------------------------

\usepackage{enumitem}

\usepackage{lettrine}

% ------------------------------

\usepackage{lastpage}

\usepackage{microtype}

% ------------------------------

\usepackage{algorithm}
\usepackage[noend]{algpseudocode}
\makeatletter
\renewcommand{\ALG@name}{Algoritmus}
\makeatother
\usepackage{amsmath}
\usepackage{bbold}
\usepackage{calc}
\usepackage{dsfont}
\usepackage{mathtools}
\usepackage{tabto}


\newcommand{\mr}[1]{\mathrm{#1}}
\newcommand{\bs}[1]{\boldsymbol{#1}}
\newcommand{\bm}[1]{\mathbf{#1}}

\newcommand{\diff}[2]{\frac{\Delta #1}{\Delta #2}}
\newcommand{\der}[2]{\frac{d #1}{d #2}}
\newcommand{\parder}[2]{\frac{\partial #1}{\partial #2}}

\newcommand{\argmax}[0]{\mr{argmax}}
\newcommand{\diag}[0]{\mr{diag}}
\newcommand{\rank}[0]{\mr{rank}}
\newcommand{\trace}[0]{\mr{tr}}

\renewcommand{\Re}{\mr{Re}}
\renewcommand{\Im}{\mr{Im}}


\theoremstyle{definition}
\newtheorem{definition}{Definícia}[section]
\newtheorem{theorem}{Veta}[section]
\newtheorem{lemma}[theorem]{Lemma}
\newtheorem{example}{Príklad}[section]
\renewcommand*{\proofname}{Dôkaz}

% ------------------------------


% -----------------------------------------------------------------------------

\def\oznacenieCelku{Kolekcia učebných textov}

% -----------------------------------------------------------------------------

\def\KUTporadoveCislo{001}

% \def\oznacenieVerzie{v1.0}
\def\oznacenieVerzie{\phantom{v1.0}}

\def\mesiacRok{február 2024}

\def\authorslabel{}






% -----------------------------------------------------------------------------

\begin{document}


% -----------------------------------------------------------------------------
% Uvodny nadpis

\noindent
\parbox[t][18mm][c]{0.3\textwidth}{%
\raisebox{-0.9\height}{%
\phantom{.}\includegraphics[height=18mm]{../../COMMONFILES/URKFEIlogo.pdf}%
}%
}%
\parbox[t][18mm][c]{0.7\textwidth}{%
\raggedleft



\sffamily
\fontsize{16pt}{18pt}
\fontseries{sbc}
\selectfont

\noindent
\textcolor[rgb]{0.75, 0.75, 0.75}{\textls[30]{\oznacenieCelku}}
}%

\noindent
\parbox[t][16mm][b]{0.5\textwidth}{%
\raggedright

\color{Gray}
\sffamily


\fontsize{8pt}{10pt}
\selectfont

\authorslabel

\fontsize{6pt}{10pt}
\selectfont

urk.fei.stuba.sk

\fontsize{12pt}{12pt}
\selectfont
\mesiacRok


}%
\parbox[t][16mm][b]{0.5\textwidth}{%
\raggedleft

\sffamily
% \color{Gray}
% \lstyle 

\fontsize{6pt}{6pt}
\selectfont

\textcolor[rgb]{0.68, 0.68, 0.68}{\oznacenieVerzie}

% \textcolor[rgb]{0.68, 0.68, 0.68}{počet strán: \pageref{LastPage}}

\fontsize{14pt}{14pt}
% \fontseries{sbc}
\selectfont

\bfseries

\includegraphics[height=12pt]{../../COMMONFILES/KUT_logo_v0.1.pdf}%
% \oznacenieCasti%
{%
% \lstyle
% \pstyle
\textls[-50]{\KUTporadoveCislo}
}%

}%

% -----------------------------------------------------------------------------




\vspace{6mm}

% ---------------------------------------------
\sffamily
\bfseries
\fontsize{18pt}{21pt}
\selectfont

\begin{flushleft}
    O základných vlastnostiach lineárnych systémov
\end{flushleft}

\bigskip

% -----------------------------------------------------------------------------
\normalsize
\normalfont
% -----------------------------------------------------------------------------



\section{Stabilita}


\subsection{Lineárny systém}
\label{Stability.LinearSystem}

Predpokladajme, že lineárny systém je opísateľný diferenciálnou rovnicou:
\begin{equation}
    y^{(n)} + a_{n - 1} y^{(n - 1)} + \cdots + a_{1} \dot{y} + a_{0} y = 0    
\end{equation}
kde $(n)$ označuje $n$-tú deriváciu podľa času. Všeobecné riešenie tejto rovnice je možné nájsť ako lineárnu kombináciu všetkých riešení v tvare:
\begin{equation}
    \label{Stability.LinearSystem.Equation:LDESolution}
    y = 
    C_{1}(t) \exp(\lambda_{1} t) + C_{2}(t) \exp(\lambda_{2} t) + \cdots + C_{n}(t) \exp(\lambda_{n} t)
\end{equation}
kde $\lambda_{i}, \ i = 1, 2, \cdots, n$ sú vlastné čísla (vo všeobecnosti sú komplexné), získame ich z charakteristickej rovnice:
\begin{equation}
    \lambda^{n} + a_{n - 1} \lambda^{n - 1} + \cdots + a_{1} \lambda + a_{0} = 0    
\end{equation} 
Koeficienty $C_i(t)$ sú vo všeobecnosti polynómy, ktorých koeficienty sú komplexné čísla. Stupeň jednotlivých polynómov sa určuje podľa násobnosti príslušných vlastných čísel. V prípade, že všetky vlastné čísla sú rozdielne, tak ide o konštanty. V prípade dvojnásobného vlastného čísla $\lambda_{i}$ koeficient $C_{i}$ bude polynóm prvého rádu a tak ďalej.

Aby všeobecné riešenie (\ref{Stability.LinearSystem.Equation:LDESolution}) bolo stabilné (konvergovalo do $0$) je podstatné, aby každá exponenciála konvergovala do $0$, koeficienty $C_i(t)$ nevplývajú na stabilitu, pretože sú to polynómy, ktoré sú dominované exponenciálnou funkciou. Každé riešenie môžeme zapísať v takomto tvare:
\begin{align}
    y &= 
    C \exp \left( \lambda t \right) = 
    C \exp \left( \Re\{\lambda\} t + i \, \Im\{\lambda\} t \right) \nonumber \\ &=  
    C \exp \left( \Re\{\lambda\} t \right) \ \left[ \cos \left( \Im\{\lambda\} t \right) + i \sin \left(\Im\{\lambda\} t \right) \right]
\end{align}
kde vidíme, že imaginárna časť vlastného čísla je argumentom funkcií kosínus a sínus. Tieto funkcie sú ohraničené, čiže imaginárna časť nemá vplyv na stabilitu, spôsobuje ale kmity v riešení. Reálna časť vlastného čísla je argumentom exponenciálnej funkcie. Exponenciálna funkcia konverguje do $0$ v prípade, že jej argument je záporný. Z toho vyplýva nasledujúce tvrdenie:
\begin{theorem}
    \label{Stability.LinearSystem.Theorem:Stability}
    Lineárny systém rádu $n$ s vlastnými číslami $\lambda_{i}, \ i = 1, 2, \cdots, n$ je
    \begin{enumerate}
        \begin{subequations}
            \item stabilný pokiaľ pre všetky vlastné čísla $\lambda_{i}$:
            \begin{equation}
                \Re\{\lambda_i\} < 0, \quad i = 1, 2, \cdots, n
            \end{equation}

            \item na hranici stability pokiaľ aspoň pre jedno vlastné číslo $\lambda_{i}$:
            \begin{equation}
                \Re\{\lambda_i\} = 0, \quad i = 1, 2, \cdots, n
            \end{equation}
            
            \item nestabilný pokiaľ aspoň pre jedno vlastné číslo $\lambda_{i}$:
            \begin{equation}
                \Re\{\lambda_i\} > 0, \quad i = 1, 2, \cdots, n
            \end{equation}
        \end{subequations}
    \end{enumerate}
\end{theorem}

V prípade stavového systému:
\begin{equation}
    \dot{\bm{x}} = \bm{A} \bm{x}
\end{equation}
môžeme všeobecné riešenie zapísať v tvare:
\begin{equation}
    \bm{x}(t) = \exp(\bm{A} (t - t_i)) \bm{x}(t_i)
\end{equation}
Rovnako platí tvrdenie (\ref{Stability.LinearSystem.Theorem:Stability}) a vlastné čísla môžeme získať z charakteristickej rovnice:
\begin{equation}
    \det \left\{ \bm{A} - \lambda \bm{I} \right\} = 0
\end{equation} 


\subsection{Stabilita vo všeobecnosti - Lyapunovova teória stability}

Vyšetrovanie stability tak ako je popísané v časti \ref{Stability.LinearSystem} platí len pre lineárne systémy. V~prípade, že chceme vyšetriť stabilitu všeobecného systému v~tvare: 
\begin{equation}
    \label{Stability.Lyapunov.Equation:System}
    \dot{\bm{x}} = \bm{f}(\bm{x})
\end{equation}
musíme aplikovať Lyapunovovu teóriu stability. Táto teória je použiteľná aj na lineárne systémy.

\begin{definition}[Stabilita podľa Lyapunova, priama metóda]
    Dynamický systém (\ref{Stability.Lyapunov.Equation:System}) je lokálne stabilný pokiaľ existuje Lyapunovova funkcia $V: \mathbb{R}^n \rightarrow \mathbb{R}$, pre ktorú platí:
    \begin{subequations}
        \begin{enumerate}
            \item V rovnovážnom stave $\bm{0}$ platí:
            \begin{equation}
                \label{Stability.Lyapunov.Definition.Equation:Equilibrium}
                V(\bm{0}) = 0 \\
            \end{equation}
            
            \item Lyapunovova funkcia je pozitívne definitná:
            \begin{equation}
                \label{Stability.Lyapunov.Definition.Equation:PositiveDefinitness}
                V(\bm{x}) > 0, \quad \bm{x} \neq \bm{0} \\
            \end{equation}
            
            \item Časová derivácia Lyapunovovej funkcie je negatívne semidefinitná:
            \begin{equation}
                \label{Stability.Lyapunov.Definition.Equation:NegativeDefinitness}
                \dot{V}(\bm{x}) = \sum_{i} \parder{V}{x^i} f^i(\bm{x}) = \nabla V^\top \bm{f}(\bm{x}) \leq 0
            \end{equation}
            V prípade asymptotickej stability musí byť negatívne definitná:
            \begin{equation}
                \label{Stability.Lyapunov.Definition.Equation:NegativeDefinitnessAsymptotical}
                \dot{V}(\bm{x}) < 0, \quad \dot{V}(\bm{0}) = 0
            \end{equation}
            V diskrétnom prípade sa derivácia nahrádza diferenciou, ostatné tvrdenia platia rovnako:
            \begin{equation}
                \label{Stability.Lyapunov.Definition.Equation:NegativeDefinitnessDiscrete}
                V(\bm{x}_{n + 1}) - V(\bm{x}_{n}) < 0
            \end{equation}

            \item Pokiaľ je $V(\bm{x})$ radiálne neohraničená:
            \begin{equation}
                ||\bm{x}|| \rightarrow \infty \ \Rightarrow \ V(\bm{x}) \rightarrow \infty
            \end{equation}
            tak hovoríme, že je systém globálne stabilný.
        \end{enumerate}
    \end{subequations}
\end{definition}


\subsection{Diskrétny systém}

Vyšetríme teraz pomocou Lyapunovovej teórie stability stabilitu lineárneho diskrétneho systému v~tvare:
\begin{equation}
    \label{Stability.Lyapunov.Discrete.Equation:StateSpaceModel}
    \bm{x}_{n + 1} = \bm{A} \bm{x}_{n}
\end{equation}

Zvolíme kandidáta na Lyapunovovu funkciu v~tvare:
\begin{equation}
    \label{Stability.Lyapunov.Discrete.Equation:LyapunovCandidate}
    V(\bm{x}_{n}) = \bm{x}_{n}^\top \bm{P} \bm{x}_{n}
\end{equation}
kde $\bm{P}$ musí byť pozitívne definitná matica, aby bola splnená podmienka (\ref{Stability.Lyapunov.Definition.Equation:PositiveDefinitness})

Ďalej musí byť splnená podmienka (\ref{Stability.Lyapunov.Definition.Equation:NegativeDefinitnessDiscrete}), dosadíme teda (\ref{Stability.Lyapunov.Discrete.Equation:LyapunovCandidate}):
\begin{equation}
    \label{Stability.Lyapunov.Discrete.Equation:LyapunovEquation1}
    \bm{x}_{n + 1}^\top \bm{P} \bm{x}_{n + 1} - \bm{x}_{n}^\top \bm{P} \bm{x}_{n} < 0
\end{equation}
a do (\ref{Stability.Lyapunov.Discrete.Equation:LyapunovEquation1}) dosadíme rovnicu systému (\ref{Stability.Lyapunov.Discrete.Equation:StateSpaceModel}) a~stavové vektory vyberieme pred zátvorku $\bm{x}_{n}$:
\begin{equation}
    \label{Stability.Lyapunov.Discrete.Equation:LyapunovEquation2}
    \bm{x}_{n}^\top \left( \bm{A}^\top \bm{P} \bm{A} - \bm{P} \right) \bm{x}_{n} < 0
\end{equation}
Rovnica (\ref{Stability.Lyapunov.Discrete.Equation:LyapunovEquation2}) musí platiť pre všetky $\bm{x}_{n}$, teda musí platiť:
\begin{equation}
    \label{Stability.Lyapunov.Discrete.Equation:LyapunovEquation3}
    \bm{A}^\top \bm{P} \bm{A} - \bm{P} < 0
\end{equation}
Nerovnicu (\ref{Stability.Lyapunov.Discrete.Equation:LyapunovEquation3}) je náročné riešiť, môžeme ju ale ekvivalentne zapísať ako rovnicu: 
\begin{equation}
    \label{Stability.Lyapunov.Discrete.Equation:LyapunovEquation}
    \bm{A}^\top \bm{P} \bm{A} - \bm{P} = -\bm{Q}
\end{equation}
kde $\bm{Q}$ je pozitívne definitná matica. Rovnica (\ref{Stability.Lyapunov.Discrete.Equation:LyapunovEquation}) sa nazýva diskrétna Lyapunovova rovnica, pokiaľ existuje riešenie $\bm{P}$, ktoré je pozitívne definitné, tak systém (\ref{Stability.Lyapunov.Discrete.Equation:StateSpaceModel}) je stabilný.


\subsection{Spojitý systém}
Pomocou Lyapunovovej teórie stability vyšetríme aj stabilitu lineárneho spojitého systému v~tvare:
\begin{equation}
    \label{Stability.Lyapunov.Continuous.Equation:StateSpaceModel}
    \dot{\bm{x}} = \bm{A} \bm{x}
\end{equation}

Zvolíme kandidáta na Lyapunovovu funkciu tak, aby bola splnená podmienka (\ref{Stability.Lyapunov.Definition.Equation:PositiveDefinitness}):
\begin{equation}
    \label{Stability.Lyapunov.Continuous.Equation:LyapunovCandidate}
    V = \bm{x}^\top \bm{P} \bm{x}
\end{equation}
kde $\bm{P}$ musí byť pozitívne definitná matica.

Vypočítame časovú deriváciu (\ref{Stability.Lyapunov.Continuous.Equation:LyapunovCandidate}) a vyšetríme podmienku (\ref{Stability.Lyapunov.Definition.Equation:NegativeDefinitnessAsymptotical}):
\begin{equation}
    \label{Stability.Lyapunov.Continuous.Equation:LyapunovEquation1}
    \dot{\bm{x}}^\top \bm{P} \bm{x} + \bm{x}^\top \bm{P} \dot{\bm{x}} < 0
\end{equation}
Do získaného výrazu (\ref{Stability.Lyapunov.Continuous.Equation:LyapunovEquation1}) dosadíme rovnicu systému (\ref{Stability.Lyapunov.Continuous.Equation:StateSpaceModel}):
\begin{equation}
    \label{Stability.Lyapunov.Continuous.Equation:LyapunovEquation2}
    \bm{x}^\top \left( \bm{A}^\top \bm{P} + \bm{P} \bm{A} \right) \bm{x} < 0
\end{equation}
Rovnica (\ref{Stability.Lyapunov.Continuous.Equation:LyapunovEquation2}) musí byť splnená pre každý stav $\bm{x}$ teda:
\begin{equation}
    \label{Stability.Lyapunov.Continuous.Equation:LyapunovEquation3}
    \bm{A}^\top \bm{P} + \bm{P} \bm{A} < 0
\end{equation}
Nerovnicu (\ref{Stability.Lyapunov.Continuous.Equation:LyapunovEquation3}) je náročné riešiť, môžeme ju ale ekvivalentne zapísať ako rovnicu: 
\begin{equation}
    \label{Stability.Lyapunov.Continuous.Equation:LyapunovEquation}
    \bm{A}^\top \bm{P} + \bm{P} \bm{A} = -\bm{Q}
\end{equation}
kde $\bm{Q}$ je pozitívne definitná matica. Rovnica (\ref{Stability.Lyapunov.Continuous.Equation:LyapunovEquation}) sa nazýva spojitá Lyapunovova rovnica, pokiaľ existuje riešenie $\bm{P}$, ktoré je pozitívne definitné, tak systém (\ref{Stability.Lyapunov.Continuous.Equation:StateSpaceModel}) je stabilný.



\end{document}
