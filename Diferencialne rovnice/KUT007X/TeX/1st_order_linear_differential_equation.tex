\documentclass[a4paper, 10pt, ]{article}

\usepackage[slovak]{babel}

% ------------------------------

\usepackage[utf8]{inputenc}
\usepackage[T1]{fontenc}

\usepackage[left=4cm,
            right=4cm,
            top=2.1cm,
            bottom=2.6cm,
            footskip=7.5mm,
            twoside,
            marginparwidth=3.0cm,
            %showframe,
            ]{geometry}

\usepackage{graphicx}
\usepackage[dvipsnames]{xcolor}
% https://en.wikibooks.org/wiki/LaTeX/Colors

% ------------------------------

\usepackage{lmodern}

\usepackage[tt={oldstyle=false,proportional=true,monowidth}]{cfr-lm}
% https://mirror.szerverem.hu/ctan/fonts/cfr-lm/doc/cfr-lm.pdf

% ------------------------------

\usepackage{amsmath}
\usepackage{amssymb}
\usepackage{amsthm}

\usepackage{booktabs}
\usepackage{multirow}
\usepackage{array}
\usepackage{dcolumn}

\usepackage{natbib}

% ------------------------------

\hyphenpenalty=6000
\tolerance=1000

\def\naT{\mathsf{T}}

% ------------------------------

\makeatletter

    \def\@seccntformat#1{\protect\makebox[0pt][r]{\csname the#1\endcsname\hspace{4mm}}}

    \def\cleardoublepage{\clearpage\if@twoside \ifodd\c@page\else
    \hbox{}
    \vspace*{\fill}
    \begin{center}
    \phantom{}
    \end{center}
    \vspace{\fill}
    \thispagestyle{empty}
    \newpage
    \if@twocolumn\hbox{}\newpage\fi\fi\fi}

    \newcommand\figcaption{\def\@captype{figure}\caption}
    \newcommand\tabcaption{\def\@captype{table}\caption}

\makeatother

% ------------------------------

\usepackage{fancyhdr}
\fancypagestyle{plain}{%
\fancyhf{} % clear all header and footer fields
% \fancyfoot[C]{\sffamily {\bfseries \thepage}\ | {\scriptsize\oznacenieCasti}}
\fancyfoot[C]{\sffamily {\bfseries \thepage}{\color{Gray}\scriptsize$\,$z$\,$\pageref{LastPage}}\ | \includegraphics[height=5pt]{./COMMONFILES/KUT_logo_v0.1.pdf}{\scriptsize\KUTporadoveCislo}}
\renewcommand{\headrulewidth}{0pt}
\renewcommand{\footrulewidth}{0pt}}
\pagestyle{plain}

% ------------------------------

\usepackage{titlesec}
\titleformat{\paragraph}[hang]{\sffamily  \bfseries}{}{0pt}{}
\titlespacing*{\paragraph}{0mm}{3mm}{1mm}
\titlespacing*{\subparagraph}{0mm}{3mm}{1mm}

\titleformat*{\section}{\sffamily\Large\bfseries}
\titleformat*{\subsection}{\sffamily\large\bfseries}
\titleformat*{\subsubsection}{\sffamily\normalsize\bfseries}


% ------------------------------

\PassOptionsToPackage{hyphens}{url}
\usepackage[pdfauthor={},
            pdftitle={},
            pdfsubject={},
            pdfkeywords={},
            % hidelinks,
            colorlinks=false,
            breaklinks,
            ]{hyperref}


% ------------------------------

\graphicspath{%
{../fig_standalone/}%
{../../PY/fig/}%
{../../ML/fig/}%
{./fig/}%
}

% ------------------------------

\usepackage{enumitem}

\usepackage{lettrine}

% ------------------------------

\usepackage{lastpage}

\usepackage{microtype}

% ------------------------------

\usepackage{algorithm}
\usepackage[noend]{algpseudocode}
\makeatletter
\renewcommand{\ALG@name}{Algoritmus}
\makeatother
\usepackage{amsmath}
\usepackage{bbold}
\usepackage{calc}
\usepackage{dsfont}
\usepackage{mathtools}
\usepackage{tabto}


\newcommand{\mr}[1]{\mathrm{#1}}
\newcommand{\bs}[1]{\boldsymbol{#1}}
\newcommand{\bm}[1]{\mathbf{#1}}

\newcommand{\diff}[2]{\frac{\Delta #1}{\Delta #2}}
\newcommand{\der}[2]{\frac{d #1}{d #2}}
\newcommand{\parder}[2]{\frac{\partial #1}{\partial #2}}

\newcommand{\argmax}[0]{\mr{argmax}}
\newcommand{\diag}[0]{\mr{diag}}
\newcommand{\rank}[0]{\mr{rank}}
\newcommand{\trace}[0]{\mr{tr}}

\renewcommand{\Re}{\mr{Re}}
\renewcommand{\Im}{\mr{Im}}


\theoremstyle{definition}
\newtheorem{definition}{Definícia}[section]
\newtheorem{theorem}{Veta}[section]
\newtheorem{lemma}[theorem]{Lemma}
\newtheorem{example}{Príklad}[section]
\renewcommand*{\proofname}{Dôkaz}

% ------------------------------


% -----------------------------------------------------------------------------

\def\oznacenieCelku{Kolekcia učebných textov}

% -----------------------------------------------------------------------------


\def\KUTporadoveCislo{00X}

\def\oznacenieVerzie{v0.9}
% \def\oznacenieVerzie{\phantom{v1.0}}

\def\mesiacRok{apríl 2024}

\def\authorslabel{RM}





% -----------------------------------------------------------------------------

\begin{document}

% -----------------------------------------------------------------------------
% Uvodny nadpis

\noindent
\parbox[t][18mm][c]{0.3\textwidth}{%
\raisebox{-0.9\height}{%
\phantom{.}\includegraphics[height=18mm]{./COMMONFILES/URKFEIlogo.pdf}%
}%
}%
\parbox[t][18mm][c]{0.7\textwidth}{%
\raggedleft

\sffamily
\fontsize{16pt}{18pt}
\fontseries{sbc}
\selectfont

\noindent
\textcolor[rgb]{0.75, 0.75, 0.75}{\textls[25]{\oznacenieCelku}}
}%

\noindent
\parbox[t][16mm][b]{0.5\textwidth}{%
\raggedright

\color{Gray}
\sffamily

\fontsize{12pt}{12pt}
\selectfont
\mesiacRok

\fontsize{6pt}{10pt}
\selectfont
github.com/PracovnyBod/KUT

\fontsize{8pt}{10pt}
\selectfont
\authorslabel




}%
\parbox[t][16mm][b]{0.5\textwidth}{%
\raggedleft

\sffamily

\fontsize{6pt}{6pt}
\selectfont

\textcolor[rgb]{0.68, 0.68, 0.68}{\oznacenieVerzie}


\fontsize{14pt}{14pt}
\selectfont

\bfseries

\includegraphics[height=12pt]{./COMMONFILES/KUT_logo_v0.1.pdf}%
{%
\textls[-50]{\KUTporadoveCislo}
}%
}%

% -----------------------------------------------------------------------------





\vspace{6mm}

% ---------------------------------------------
\sffamily
\bfseries
\fontsize{18pt}{21pt}
\selectfont

\begin{flushleft}
	O riešení lineárnych diferenciálnych \\ rovníc 1.~rádu
\end{flushleft}

\bigskip

% -----------------------------------------------------------------------------
\normalsize
\normalfont
% -----------------------------------------------------------------------------



\section{Lineárna diferenciálna rovnica 1. rádu}

Diferenciálnu rovnicu prvého rádu môžeme zapísať vo všeobecnom tvare:
\begin{equation}
    \label{1stOderLDE.Equation:1stOrderLinearDifferentialEquation}
    \dot{x}(t) = a(t) x(t) + b(t) u(t)
\end{equation}
kde $a(t)$ a $b(t)$ sú koeficienty, ktoré sú premenlivé (závisia od premennej $t$ v ponímaní modelovania systémov môže predstavovať čas) a $u(t)$ je funkcia závislá od $t$. V prípade modelovania systémov predstavuje vstup do systému. Cieľom riešenia diferenciálnej rovnice (\ref{1stOderLDE.Equation:1stOrderLinearDifferentialEquation}) je nájsť také funkcie $x(t)$, ktoré túto rovnicu spĺňajú. Opäť v prípade modelovania systémov $x(t)$ predstavuje stav systému.

Keďže riešime lineárnu diferenciálnu rovnicu jej všeobecné riešenie môžeme nájsť poskladaním čiastkových riešení:
\begin{equation}
    \label{1stOderLDE.Equation:SolutionLinearCombination}
    x(t) = x_h(t) + x_p(t)
\end{equation}
kde $x_h$ sa nazýva homogénne riešenie a získame ho riešením zhomogenizovanej rovnice. Riešenie $x_p$ je partikulárne riešenie, ktoré môžeme získať metódou variácie konštánt.

Najskôr získame homogénne riešenie. Rovnicu (\ref{1stOderLDE.Equation:1stOrderLinearDifferentialEquation}) zhomogenizujeme tým, že uvažujeme $u(t) = 0$:
\begin{equation}
    \label{1stOderLDE.Equation:1stOrderLinearDifferentialEquationHomogeneous}
    \dot{x}_h(t) = a(t) x_h(t)
\end{equation}

\noindent Zhomogenizovaná rovnica (\ref{1stOderLDE.Equation:1stOrderLinearDifferentialEquationHomogeneous}) je separovateľná:
\begin{equation}
    \label{1stOderLDE.Equation:1stOrderLinearDifferentialEquationHomogeneousSeparated}
    \der{x_h(t)}{t} = a(t) x_h(t) \longrightarrow \frac{dx(t)}{x_h(t)} = a(t)dt
\end{equation}

\noindent Následne integrujeme obidve strany rovnice (\ref{1stOderLDE.Equation:1stOrderLinearDifferentialEquationHomogeneousSeparated}):
\begin{equation}
    \label{1stOrderLDE.Equation:HomogeneousIntegral}
    \ln \left( |x_h(t_f)| \right) + K = \int_{t_i}^{t_f} a(t) \, dt
\end{equation}
Integráciu realizujeme na intervale $\left< t_i, t_f \right>$, zaviedli sme konštantu $K$, ktorá vyjadruje stav v $t_i$.

Vyjadríme $x_h(t_f)$:
\begin{equation}
    x_h(t_f) = \exp(-K) \exp \left( \int_{t_i}^{t_f} a(t) \, dt \right)
\end{equation}
keďže $K$ je konštanta a rovnako aj ''$\exp$'' tak môžeme zaviesť novú konštantu $L = \exp(-K)$, takéto premenovávanie konštánt budeme využívať často.

Riešenie zhomogenizovanej rovnice (\ref{1stOderLDE.Equation:1stOrderLinearDifferentialEquationHomogeneous}) má teda tvar:
\begin{equation}
    \label{1stOderLDE.Equation:HomogeneousSolution}
    x_h(t_f) = L \exp \left( \int_{t_i}^{t_f} a(t) \, dt \right)
\end{equation}

Teraz prejdeme na vyjadrenie partikulárneho riešenia. Predpokladáme, že bude v rovnakom tvare ako homogénne riešenie (\ref{1stOderLDE.Equation:HomogeneousSolution}):
\begin{equation}
    \label{1stOderLDE.Equation:ParticularSolutionForm}
    x_p(t) = L(t) \exp \left( \int_{t_i}^{t} a(\tau) \, d\tau \right)
\end{equation}
Avšak, pripustíme, že ''konštanta'' $L$ nie je konštantou ale je variabilná (variácia konštánt). Táto variabilita má zohľadniť vplyv $u$, ktorý sme v homogénnom riešení vypustili.

Partikulárne riešenie (\ref{1stOderLDE.Equation:ParticularSolutionForm}) zderivujeme:
\begin{equation}
    \dot{x}_p(t) = 
    \dot{L}(t) \exp \left( \int_{t_i}^{t} a(\tau) \, d\tau \right) + 
    L(t) a(t) \exp \left( \int_{t_i}^{t} a(\tau) \, d\tau \right)
\end{equation}
a dosadíme do (\ref{1stOderLDE.Equation:1stOrderLinearDifferentialEquation}). Po úpravách dostaneme:
\begin{equation}
    \dot{L}(t)  = 
    \exp \left( -\int_{t_i}^{t} a(\tau) \, d\tau \right) b(t) u(t)
\end{equation}
a po integrácii:
\begin{equation}
    \label{1stOderLDE.Equation:ConstantVariation}
    L(t_f) + M = \int_{t_i}^{t_f} \exp \left( -\int_{t_i}^{t} a(\tau) \, d\tau \right) b(t) u(t) \, dt
\end{equation}

\noindent Nakoniec dosadením (\ref{1stOderLDE.Equation:ConstantVariation}) do (\ref{1stOderLDE.Equation:ParticularSolutionForm}) ešte vyjadríme partikulárne riešenie $x_p$ v bode $t_f$:
\begin{align}
    \label{1stOderLDE.Equation:ParticularSolution}
    \nonumber
    x_p(t_f) &= 
    N \exp \left( \int_{t_i}^{t_f} a(\tau) \, d\tau \right) \\ &+ 
    \int_{t_i}^{t_f} \exp \left( \int_{t_i}^{t_f} a(\tau) \, d\tau - \int_{t_i}^{t} a(\tau) \, d\tau \right) b(t) u(t) \, dt \\ &= 
    N \exp \left( \int_{t_i}^{t_f} a(\tau) \, d\tau \right) \\ &+ 
    \int_{t_i}^{t_f} \exp \left( \int_{t}^{t_f} a(\tau) \, d\tau \right) b(t) u(t) \, dt
\end{align}
kde $N$ je konštanta $N = -M$, integrály sme zjednotili na základe prepdokladu, že $t \in \left< t_i, t_f \right>$.

Získané homogénne (\ref{1stOderLDE.Equation:HomogeneousSolution}) a partikulárne (\ref{1stOderLDE.Equation:ParticularSolution}) riešenie teraz sčítame podľa (\ref{1stOderLDE.Equation:SolutionLinearCombination}):
\begin{equation}
    x(t_f) = 
    C \exp \left( \int_{t_i}^{t_f} a(t) \, dt \right) + 
    \int_{t_i}^{t_f} \exp \left( \int_{t}^{t_f} a(\tau) \, d\tau \right) b(t) u(t) \, dt
\end{equation}
kde sme zaviedli $C = L - N$, pretože všetko sú to konštanty a môžeme ich ľubovoľne premenovať. Pokiaľ ešte poznáme začiatočné podmienky $x(t_i)$, tak môžeme tvrdiť $C = x(t_i)$ a konečné riešenie lineárnej diferenciálnej rovnice 1.~rádu má tvar:
\begin{equation}
    x(t_f) = 
    x(t_i) \exp \left( \int_{t_i}^{t_f} a(t) \, dt \right) + 
    \int_{t_i}^{t_f} \exp \left( \int_{t}^{t_f} a(\tau) \, d\tau \right) b(t) u(t) \, dt 
\end{equation}



\section{Sústava lineárnych diferenciálnych rovníc 1. rádu}

Rovnako môžeme postupovať aj v prípade sústavy diferenciálnych rovníc 1.~rádu:
\begin{equation}
    \label{SystemLDE:StateSpace}
    \dot{\bm{x}}(t) = \bm{A} \bm{x}(t) + \bm{B} \bm{u}(t)
\end{equation}
Tento tvar je zámerne zhodný s tvarom stavového modelu. Matice $\bm{A}$ a $\bm{B}$ sú konštantné, pretože v prípade, že by boli časovo premenlivé tak neexistuje analytické riešenie. A teda analytické riešenie sústavy diferenciálnych rovníc 1. rádu (\ref{SystemLDE:StateSpace}) má tvar:
\begin{equation}
    \bm{x}(t_f) = 
    \bm{x}(t_i) \exp \left( \bm{A} (t_f - t_i) \right) + 
    \int_{t_i}^{t_f} \exp \left( \bm{A} (t_f - t) \right) \bm{B} \bm{u}(t) \, dt 
\end{equation}

Maticový exponent vieme rozpísať do mocninového radu:
\begin{equation}
    \exp(\bm{M}) = \sum_{n = 0}^{\infty} \frac{1}{n!}\bm{M}^n
\end{equation}


\end{document}
