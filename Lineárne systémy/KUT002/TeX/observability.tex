\documentclass[a4paper, 10pt, ]{article}

\usepackage[slovak]{babel}

% ------------------------------

\usepackage[utf8]{inputenc}
\usepackage[T1]{fontenc}

\usepackage[left=4cm,
            right=4cm,
            top=2.1cm,
            bottom=2.6cm,
            footskip=7.5mm,
            twoside,
            marginparwidth=3.0cm,
            %showframe,
            ]{geometry}

\usepackage{graphicx}
\usepackage[dvipsnames]{xcolor}
% https://en.wikibooks.org/wiki/LaTeX/Colors

% ------------------------------

\usepackage{lmodern}

\usepackage[tt={oldstyle=false,proportional=true,monowidth}]{cfr-lm}
% https://mirror.szerverem.hu/ctan/fonts/cfr-lm/doc/cfr-lm.pdf

% ------------------------------

\usepackage{amsmath}
\usepackage{amssymb}
\usepackage{amsthm}

\usepackage{booktabs}
\usepackage{multirow}
\usepackage{array}
\usepackage{dcolumn}

\usepackage{natbib}

% ------------------------------

\hyphenpenalty=6000
\tolerance=1000

\def\naT{\mathsf{T}}

% ------------------------------

\makeatletter

    \def\@seccntformat#1{\protect\makebox[0pt][r]{\csname the#1\endcsname\hspace{4mm}}}

    \def\cleardoublepage{\clearpage\if@twoside \ifodd\c@page\else
    \hbox{}
    \vspace*{\fill}
    \begin{center}
    \phantom{}
    \end{center}
    \vspace{\fill}
    \thispagestyle{empty}
    \newpage
    \if@twocolumn\hbox{}\newpage\fi\fi\fi}

    \newcommand\figcaption{\def\@captype{figure}\caption}
    \newcommand\tabcaption{\def\@captype{table}\caption}

\makeatother

% ------------------------------

\usepackage{fancyhdr}
\fancypagestyle{plain}{%
\fancyhf{} % clear all header and footer fields
% \fancyfoot[C]{\sffamily {\bfseries \thepage}\ | {\scriptsize\oznacenieCasti}}
\fancyfoot[C]{\sffamily {\bfseries \thepage}{\color{Gray}\scriptsize$\,$z$\,$\pageref{LastPage}}\ | \includegraphics[height=5pt]{./COMMONFILES/KUT_logo_v0.1.pdf}{\scriptsize\KUTporadoveCislo}}
\renewcommand{\headrulewidth}{0pt}
\renewcommand{\footrulewidth}{0pt}}
\pagestyle{plain}

% ------------------------------

\usepackage{titlesec}
\titleformat{\paragraph}[hang]{\sffamily  \bfseries}{}{0pt}{}
\titlespacing*{\paragraph}{0mm}{3mm}{1mm}
\titlespacing*{\subparagraph}{0mm}{3mm}{1mm}

\titleformat*{\section}{\sffamily\Large\bfseries}
\titleformat*{\subsection}{\sffamily\large\bfseries}
\titleformat*{\subsubsection}{\sffamily\normalsize\bfseries}


% ------------------------------

\PassOptionsToPackage{hyphens}{url}
\usepackage[pdfauthor={},
            pdftitle={},
            pdfsubject={},
            pdfkeywords={},
            % hidelinks,
            colorlinks=false,
            breaklinks,
            ]{hyperref}


% ------------------------------

\graphicspath{%
{../fig_standalone/}%
{../../PY/fig/}%
{../../ML/fig/}%
{./fig/}%
}

% ------------------------------

\usepackage{enumitem}

\usepackage{lettrine}

% ------------------------------

\usepackage{lastpage}

\usepackage{microtype}

% ------------------------------

\usepackage{algorithm}
\usepackage[noend]{algpseudocode}
\makeatletter
\renewcommand{\ALG@name}{Algoritmus}
\makeatother
\usepackage{amsmath}
\usepackage{bbold}
\usepackage{calc}
\usepackage{dsfont}
\usepackage{mathtools}
\usepackage{tabto}


\newcommand{\mr}[1]{\mathrm{#1}}
\newcommand{\bs}[1]{\boldsymbol{#1}}
\newcommand{\bm}[1]{\mathbf{#1}}

\newcommand{\diff}[2]{\frac{\Delta #1}{\Delta #2}}
\newcommand{\der}[2]{\frac{d #1}{d #2}}
\newcommand{\parder}[2]{\frac{\partial #1}{\partial #2}}

\newcommand{\argmax}[0]{\mr{argmax}}
\newcommand{\diag}[0]{\mr{diag}}
\newcommand{\rank}[0]{\mr{rank}}
\newcommand{\trace}[0]{\mr{tr}}

\renewcommand{\Re}{\mr{Re}}
\renewcommand{\Im}{\mr{Im}}


\theoremstyle{definition}
\newtheorem{definition}{Definícia}[section]
\newtheorem{theorem}{Veta}[section]
\newtheorem{lemma}[theorem]{Lemma}
\newtheorem{example}{Príklad}[section]
\renewcommand*{\proofname}{Dôkaz}

% ------------------------------


% -----------------------------------------------------------------------------

\def\oznacenieCelku{Kolekcia učebných textov}

% -----------------------------------------------------------------------------

\def\KUTporadoveCislo{002}

% \def\oznacenieVerzie{v1.0}
\def\oznacenieVerzie{\phantom{v1.0}}

\def\mesiacRok{február 2024}

\def\authorslabel{}






% -----------------------------------------------------------------------------

\begin{document}


% -----------------------------------------------------------------------------
% Uvodny nadpis

\noindent
\parbox[t][18mm][c]{0.3\textwidth}{%
\raisebox{-0.9\height}{%
\phantom{.}\includegraphics[height=18mm]{./COMMONFILES/URKFEIlogo.pdf}%
}%
}%
\parbox[t][18mm][c]{0.7\textwidth}{%
\raggedleft



\sffamily
\fontsize{16pt}{18pt}
\fontseries{sbc}
\selectfont

\noindent
\textcolor[rgb]{0.75, 0.75, 0.75}{\textls[30]{\oznacenieCelku}}
}%

\noindent
\parbox[t][16mm][b]{0.5\textwidth}{%
\raggedright

\color{Gray}
\sffamily


\fontsize{8pt}{10pt}
\selectfont

\authorslabel

\fontsize{6pt}{10pt}
\selectfont

urk.fei.stuba.sk

\fontsize{12pt}{12pt}
\selectfont
\mesiacRok


}%
\parbox[t][16mm][b]{0.5\textwidth}{%
\raggedleft

\sffamily
% \color{Gray}
% \lstyle 

\fontsize{6pt}{6pt}
\selectfont

\textcolor[rgb]{0.68, 0.68, 0.68}{\oznacenieVerzie}

% \textcolor[rgb]{0.68, 0.68, 0.68}{počet strán: \pageref{LastPage}}

\fontsize{14pt}{14pt}
% \fontseries{sbc}
\selectfont

\bfseries

\includegraphics[height=12pt]{./COMMONFILES/KUT_logo_v0.1.pdf}%
% \oznacenieCasti%
{%
% \lstyle
% \pstyle
\textls[-50]{\KUTporadoveCislo}
}%

}%

% -----------------------------------------------------------------------------




\vspace{6mm}

% ---------------------------------------------
\sffamily
\bfseries
\fontsize{18pt}{21pt}
\selectfont

\begin{flushleft}
    O základných vlastnostiach lineárnych systémov
\end{flushleft}

\bigskip

% -----------------------------------------------------------------------------
\normalsize
\normalfont
% -----------------------------------------------------------------------------



\section{Pozorovateľnosť}

Pozorovateľnosť je mierou ako dobre je možné určiť vnútorný stavový vektor na základe výstupov systému. Systém je pozorovateľný, pokiaľ aktuálny stavový vektor je možné určiť len na základe nameraných výstupov systému. 


\subsection{Diskrétny systém}

V nasledujúcej časti odvodíme podmienku pozorovateľnosti pre diskrétny lineárny systém:
\begin{equation}
    \label{Observability.Discrete.Equation:StateSpaceModel}
    \bm{x}_{n + 1} = \bm{A} \bm{x}_{n}
\end{equation}
kde $\bm{x}_{n}$ je vnútorný stav, ktorý sa budeme snažiť určiť na základe výstupu:
\begin{equation}
    \label{Observability.Discrete.Equation:Output}
    \bm{y}_n = \bm{C} \bm{x}_{n}
\end{equation}

Na určenie $n$ prvkového stavového vektora potrebujeme práve $n$ meraní výstupu. Rovnicu výstupu (\ref{Observability.Discrete.Equation:Output}) môžeme teda rozpísať do sústavy $n$ rovníc:
\begin{subequations}
    \label{Observability.Discrete.Equation:OutputSystem}
    \begin{align}
        \bm{y}_0       &= \bm{C} \bm{x}_{0}                \\
        \bm{y}_1       &= \bm{C} \bm{A} \bm{x}_{0}         \\
        &\vdotswithin{=}                                   \\
        \bm{y}_{n - 1} &= \bm{C} \bm{A}^{n - 1} \bm{x}_{0}
    \end{align}
\end{subequations}
Stačí určiť stav $\bm{x}_0$, pretože na základe neho vývojom systému potom vieme dopočítať všetky možné vnútorné stavy. Pridanie ďalšieho merania už neprinesie žiadnu novú informáciu, pretože na základe Cayleyho-Hamiltonovho teorému môžeme $n$-tú mocninu matice napísať ako lineárnu kombináciu jej predošlých $n - 1$ mocnín:
\begin{equation}
    \bm{A}^n = \sum_{i = 0}^{n - 1} \alpha_{i} \bm{A}^{i}
\end{equation}
čiže nová informácia by bola lineárne závislá od ostatných, to znamená, že už je zahrnutá a nemá zmysel ju znova vniesť do rovníc.

Systém rovníc (\ref{Observability.Discrete.Equation:OutputSystem}) môžeme tiež zapísať maticovo:
\begin{equation}
    \label{Observability.Discrete.Equation:OutputSystemMatrix}
    \begin{bmatrix}
        \bm{y}_{0}     \\
        \bm{y}_{1}     \\
        \vdots         \\
        \bm{y}_{n - 1} \\
    \end{bmatrix} = 
    \underbrace{
        \begin{bmatrix}
            \bm{C}                \\
            \bm{C} \bm{A}         \\
            \vdots                \\
            \bm{C} \bm{A}^{n - 1} \\
        \end{bmatrix}
    }_{\bs{\mathcal{O}}}
    \bm{x}_0
\end{equation}
Snažíme sa vyriešiť systém (\ref{Observability.Discrete.Equation:OutputSystemMatrix}) vzhľadom na $\bm{x}_{0}$, kde $\bs{\mathcal{O}}$ označuje maticu pozorovateľnosti s~rozmermi $mn \times n$, kde $m$ je počet výstupov. Riešenie bude existovať len vtedy ak matica pozorovateľnosti bude mať plnú hodnosť.

\begin{theorem}
    Diskrétny lineárny systém (\ref{Observability.Discrete.Equation:StateSpaceModel}) je pozorovateľný pokiaľ matica pozorovateľnosti $\bs{\mathcal{O}}$ má plnú hodnosť:
    \begin{equation}
        \label{Observability.Discrete.Equation:ObservabilityTheorem}
        \rank \left\{ \bs{\mathcal{O}} \right\} = 
        \rank \left\{
            \begin{bmatrix}
                \bm{C}                \\
                \bm{C} \bm{A}         \\
                \vdots                \\
                \bm{C} \bm{A}^{n - 1} \\
            \end{bmatrix}
        \right\} = 
        n
    \end{equation}
\end{theorem}


\subsection{Spojitý systém}

V nasledujúcej časti odvodíme podmienku pozorovateľnosti pre diskrétny lineárny systém:
\begin{equation}
    \label{Observability.Continuous.Equation:StateSpaceModel}
    \dot{\bm{x}} = \bm{A} \bm{x}
\end{equation}
kde $\bm{x}$ je vnútorný stav, ktorý sa budeme snažiť určiť na základe výstupu:
\begin{equation}
    \label{Observability.Continuous.Equation:Output}
    \bm{y} = \bm{C} \bm{x}
\end{equation}
Diferenciálna rovnica (\ref{Observability.Continuous.Equation:StateSpaceModel}) má riešenie v tvare:
\begin{equation}
    \bm{x}(t) = \exp \left( \bm{A} (t - t_i) \right)
\end{equation}
čiže opäť je postačujúce poznať stav v začiatočnom čase $t_i$.

Predpokladáme, že poznáme výstup systému $\bm{y}$ a jeho $n - 1$ derivácií:
\begin{subequations}
    \label{Observability.Continuous.Equation:OutputSystem}
    \begin{align}
        \dot{\bm{y}}(t_i)     &= \bm{C} \bm{x}(t_i)                \\
        \ddot{\bm{y}}(t_i)    &= \bm{C} \bm{A} \bm{x}(t_i)         \\
        &\vdotswithin{=}                                           \\
        \bm{y}^{(n - 1)}(t_i) &= \bm{C} \bm{A}^{n - 1} \bm{x}(t_i)
    \end{align}
\end{subequations}
Znova platí argument, že na $n$ prvkov stavu potrebujeme práve $n$ informácií, novo pridaná informácia by bola lineárne závislá od ostatných. Snažíme sa teda vyriešiť sústavu rovníc:
\begin{equation}
    \label{Observability.Continuous.Equation:OutputSystemMatrix}
    \begin{bmatrix}
        \dot{\bm{y}}(t_i)     \\
        \ddot{\bm{y}}(t_i)    \\
        \vdots                \\
        \bm{y}^{(n - 1)}(t_i)
    \end{bmatrix} = 
    \underbrace{
        \begin{bmatrix}
            \bm{C}                \\
            \bm{C} \bm{A}         \\
            \vdots                \\
            \bm{C} \bm{A}^{n - 1} \\
        \end{bmatrix}
    }_{\bs{\mathcal{O}}}
    \bm{x}(t_i)
\end{equation}
ktorá je riešiteľná len vtedy, keď matica pozorovateľnosti $\bs{\mathcal{O}}$, s~rozmermi $mn \times n$, kde $m$ je počet výstupov, má plnú hodnosť.

\begin{theorem}
    Spojitý lineárny systém (\ref{Observability.Continuous.Equation:StateSpaceModel}) je pozorovateľný pokiaľ matica pozorovateľnosti $\bs{\mathcal{O}}$ má plnú hodnosť:
    \begin{equation}
        \label{Observability.Continuous.Equation:ObservabilityTheorem}
        \rank \left\{ \bs{\mathcal{O}} \right\} = 
        \rank \left\{
            \begin{bmatrix}
                \bm{C}                \\
                \bm{C} \bm{A}         \\
                \vdots                \\
                \bm{C} \bm{A}^{n - 1} \\
            \end{bmatrix}
        \right\} = 
        n
    \end{equation}
\end{theorem}

Ako sa dalo očakávať pre diskrétny aj spojitý lineárny systém je podmienka pozorovateľnosti definovaná v~rovnakom tvare.


\end{document}
