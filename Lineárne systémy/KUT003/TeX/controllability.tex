\documentclass[a4paper, 10pt, ]{article}

\usepackage[slovak]{babel}

% ------------------------------

\usepackage[utf8]{inputenc}
\usepackage[T1]{fontenc}

\usepackage[left=4cm,
            right=4cm,
            top=2.1cm,
            bottom=2.6cm,
            footskip=7.5mm,
            twoside,
            marginparwidth=3.0cm,
            %showframe,
            ]{geometry}

\usepackage{graphicx}
\usepackage[dvipsnames]{xcolor}
% https://en.wikibooks.org/wiki/LaTeX/Colors

% ------------------------------

\usepackage{lmodern}

\usepackage[tt={oldstyle=false,proportional=true,monowidth}]{cfr-lm}
% https://mirror.szerverem.hu/ctan/fonts/cfr-lm/doc/cfr-lm.pdf

% ------------------------------

\usepackage{amsmath}
\usepackage{amssymb}
\usepackage{amsthm}

\usepackage{booktabs}
\usepackage{multirow}
\usepackage{array}
\usepackage{dcolumn}

\usepackage{natbib}

% ------------------------------

\hyphenpenalty=6000
\tolerance=1000

\def\naT{\mathsf{T}}

% ------------------------------

\makeatletter

    \def\@seccntformat#1{\protect\makebox[0pt][r]{\csname the#1\endcsname\hspace{4mm}}}

    \def\cleardoublepage{\clearpage\if@twoside \ifodd\c@page\else
    \hbox{}
    \vspace*{\fill}
    \begin{center}
    \phantom{}
    \end{center}
    \vspace{\fill}
    \thispagestyle{empty}
    \newpage
    \if@twocolumn\hbox{}\newpage\fi\fi\fi}

    \newcommand\figcaption{\def\@captype{figure}\caption}
    \newcommand\tabcaption{\def\@captype{table}\caption}

\makeatother

% ------------------------------

\usepackage{fancyhdr}
\fancypagestyle{plain}{%
\fancyhf{} % clear all header and footer fields
% \fancyfoot[C]{\sffamily {\bfseries \thepage}\ | {\scriptsize\oznacenieCasti}}
\fancyfoot[C]{\sffamily {\bfseries \thepage}{\color{Gray}\scriptsize$\,$z$\,$\pageref{LastPage}}\ | \includegraphics[height=5pt]{./COMMONFILES/KUT_logo_v0.1.pdf}{\scriptsize\KUTporadoveCislo}}
\renewcommand{\headrulewidth}{0pt}
\renewcommand{\footrulewidth}{0pt}}
\pagestyle{plain}

% ------------------------------

\usepackage{titlesec}
\titleformat{\paragraph}[hang]{\sffamily  \bfseries}{}{0pt}{}
\titlespacing*{\paragraph}{0mm}{3mm}{1mm}
\titlespacing*{\subparagraph}{0mm}{3mm}{1mm}

\titleformat*{\section}{\sffamily\Large\bfseries}
\titleformat*{\subsection}{\sffamily\large\bfseries}
\titleformat*{\subsubsection}{\sffamily\normalsize\bfseries}


% ------------------------------

\PassOptionsToPackage{hyphens}{url}
\usepackage[pdfauthor={},
            pdftitle={},
            pdfsubject={},
            pdfkeywords={},
            % hidelinks,
            colorlinks=false,
            breaklinks,
            ]{hyperref}


% ------------------------------

\graphicspath{%
{../fig_standalone/}%
{../../PY/fig/}%
{../../ML/fig/}%
{./fig/}%
}

% ------------------------------

\usepackage{enumitem}

\usepackage{lettrine}

% ------------------------------

\usepackage{lastpage}

\usepackage{microtype}

% ------------------------------

\usepackage{algorithm}
\usepackage[noend]{algpseudocode}
\makeatletter
\renewcommand{\ALG@name}{Algoritmus}
\makeatother
\usepackage{amsmath}
\usepackage{bbold}
\usepackage{calc}
\usepackage{dsfont}
\usepackage{mathtools}
\usepackage{tabto}


\newcommand{\mr}[1]{\mathrm{#1}}
\newcommand{\bs}[1]{\boldsymbol{#1}}
\newcommand{\bm}[1]{\mathbf{#1}}

\newcommand{\diff}[2]{\frac{\Delta #1}{\Delta #2}}
\newcommand{\der}[2]{\frac{d #1}{d #2}}
\newcommand{\parder}[2]{\frac{\partial #1}{\partial #2}}

\newcommand{\argmax}[0]{\mr{argmax}}
\newcommand{\diag}[0]{\mr{diag}}
\newcommand{\rank}[0]{\mr{rank}}
\newcommand{\trace}[0]{\mr{tr}}

\renewcommand{\Re}{\mr{Re}}
\renewcommand{\Im}{\mr{Im}}


\theoremstyle{definition}
\newtheorem{definition}{Definícia}[section]
\newtheorem{theorem}{Veta}[section]
\newtheorem{lemma}[theorem]{Lemma}
\newtheorem{example}{Príklad}[section]
\renewcommand*{\proofname}{Dôkaz}

% ------------------------------


% -----------------------------------------------------------------------------

\def\oznacenieCelku{Kolekcia učebných textov}

% -----------------------------------------------------------------------------

\def\KUTporadoveCislo{003}

% \def\oznacenieVerzie{v1.0}
\def\oznacenieVerzie{\phantom{v1.0}}

\def\mesiacRok{február 2024}

\def\authorslabel{}






% -----------------------------------------------------------------------------

\begin{document}


% -----------------------------------------------------------------------------
% Uvodny nadpis

\noindent
\parbox[t][18mm][c]{0.3\textwidth}{%
\raisebox{-0.9\height}{%
\phantom{.}\includegraphics[height=18mm]{../../COMMONFILES/URKFEIlogo.pdf}%
}%
}%
\parbox[t][18mm][c]{0.7\textwidth}{%
\raggedleft



\sffamily
\fontsize{16pt}{18pt}
\fontseries{sbc}
\selectfont

\noindent
\textcolor[rgb]{0.75, 0.75, 0.75}{\textls[30]{\oznacenieCelku}}
}%

\noindent
\parbox[t][16mm][b]{0.5\textwidth}{%
\raggedright

\color{Gray}
\sffamily


\fontsize{8pt}{10pt}
\selectfont

\authorslabel

\fontsize{6pt}{10pt}
\selectfont

urk.fei.stuba.sk

\fontsize{12pt}{12pt}
\selectfont
\mesiacRok


}%
\parbox[t][16mm][b]{0.5\textwidth}{%
\raggedleft

\sffamily
% \color{Gray}
% \lstyle 

\fontsize{6pt}{6pt}
\selectfont

\textcolor[rgb]{0.68, 0.68, 0.68}{\oznacenieVerzie}

% \textcolor[rgb]{0.68, 0.68, 0.68}{počet strán: \pageref{LastPage}}

\fontsize{14pt}{14pt}
% \fontseries{sbc}
\selectfont

\bfseries

\includegraphics[height=12pt]{../../COMMONFILES/KUT_logo_v0.1.pdf}%
% \oznacenieCasti%
{%
% \lstyle
% \pstyle
\textls[-50]{\KUTporadoveCislo}
}%

}%

% -----------------------------------------------------------------------------




\vspace{6mm}

% ---------------------------------------------
\sffamily
\bfseries
\fontsize{18pt}{21pt}
\selectfont

\begin{flushleft}
    O základných vlastnostiach lineárnych systémov
\end{flushleft}

\bigskip

% -----------------------------------------------------------------------------
\normalsize
\normalfont
% -----------------------------------------------------------------------------



\section{Riaditeľnosť}

Riaditeľnosť je vlastnosť dynamických systémov dosiahnuť konečný stav zo začiatočného stavu v~konečnom čase. Hovoríme, že systém je riaditeľný pokiaľ existuje taká postupnosť vstupov, ktorá zabezpečí, že systém sa dostane z ľubovoľného začiatočného stavu do ľubovoľného konečného stavu v~konečnom čase.

\subsection{Diskrétny systém}

Diskrétny lineárny systém je opísaný stavovou rovnicou:
\begin{equation}
    \label{Controllability.Discrete.Equation:StateSpaceModel}
    \bm{x}_{n + 1} = \bm{A} \bm{x}_{n} + \bm{B} \bm{u}_{n}
\end{equation}
Vývoj systému (\ref{Controllability.Discrete.Equation:StateSpaceModel}) môžeme rozpísať do sústavy rovníc:
\begin{subequations}
    \begin{align}
        \bm{x}_{1} &= \bm{A} \bm{x}_{0} + \bm{B} \bm{u}_{0} \\
        \bm{x}_{2} &= \bm{A}^2 \bm{x}_{0} + \bm{A} \bm{B} \bm{u}_{0} + \bm{B} \bm{u}_{1} \\
        &\vdotswithin{=} \nonumber \\
        \label{Controllability.Discrete.Equation:ControllabilityEquationFinalState}
        \bm{x}_{n} &= \bm{A}^{n} \bm{x}_{0} + \bm{A}^{n - 1} \bm{B} \bm{u}_{0} + \cdots + \bm{A} \bm{B} \bm{u}_{n - 2} + \bm{B} \bm{u}_{n - 1}
    \end{align}
\end{subequations}
Rovnica \ref{Controllability.Discrete.Equation:ControllabilityEquationFinalState} opisuje postupnosť ako dosiahnuť konečný stav $\bm{x}_n$, kde $n$ označuje celkový počet stavov. Rovnako ju môžeme zapísať aj v maticovom tvare:
\begin{equation}
    \label{Controllability.Discrete.Equation:ControllabilityEquation}
    \bm{x}_{n} - \bm{A}^{n} \bm{x}_{0} = 
    \underbrace{
        \begin{bmatrix}
            \bm{B} & \bm{A} \bm{B} & \cdots & \bm{A}^{n - 1} \bm{B}
        \end{bmatrix}
    }_{\bs{\mathcal{C}}}
    \begin{bmatrix}
        \bm{u}_{n - 1} \\
        \bm{u}_{n - 2} \\
        \vdots         \\
        \bm{u}_{0}     \\
    \end{bmatrix}
\end{equation}
kde $\bs{\mathcal{C}}$ označuje maticu riaditeľnosti. Matica $\bs{\mathcal{C}}$ má rozmery $n \times mn$, kde $m$ je počet vstupov (rozmer vektora $\bm{u}$). Rovnicu (\ref{Controllability.Discrete.Equation:ControllabilityEquation}) riešime pre vstupy $\bm{u}_{i}$ a~riešenie existuje pokiaľ matica riaditeľnosti $\bs{\mathcal{C}}$ má plnú hodnosť.

\begin{theorem}
    Diskrétny lineárny systém (\ref{Controllability.Discrete.Equation:StateSpaceModel}) je riaditeľný pokiaľ matica riaditeľnosti má plnú hodnosť:
    \begin{equation}
        \label{Controllability.Discrete.Equation:ControllabilityTheorem}
        \rank \left\{ \bs{\mathcal{C}} \right\} = 
        \rank \left\{
            \begin{bmatrix}
                \bm{B} & \bm{A} \bm{B} & \cdots & \bm{A}^{n - 1} \bm{B}
            \end{bmatrix}
        \right\} = 
        n
    \end{equation}
\end{theorem}



\subsection{Spojitý systém}

V prípade spojitého lineárneho systému v~stavovom opise:
\begin{equation}
    \label{Controllability.Continuous.Equation:StateSpaceModel}
    \dot{\bm{x}} = \bm{A} \bm{x} + \bm{B} \bm{u}
\end{equation}
môžeme riešenie zo začiatočného stavu $\bm{x}(t_i)$ do koncového stavu $\bm{x}(t_f)$ nájsť v~tvare:
\begin{equation}
    \label{Controllability.Continuous.Equation:StateSpaceModelSolution}
    \bm{x}(t_f) = \exp \left( \bm{A} (t_f - t_i) \right) + \int_{t_i}^{t_f} \exp(\bm{A} (t_f - \tau)) \bm{B} \bm{u}(\tau) d\tau
\end{equation}
alebo:
\begin{equation}
    \label{Controllability.Continuous.Equation:StateSpaceModelSolutionModified}
    \exp(-\bm{A} t_f) \bm{x}(t_f) - \exp(-\bm{A} t_i) \bm{x}(t_i) = \int_{t_i}^{t_f} \exp(-\bm{A} \tau) \bm{B} \bm{u}(\tau) d\tau
\end{equation}
Maticový exponent môžeme vyjadriť ako nekonečný rad:
\begin{equation}
    \label{Controllability.Continuous.Equation:MatrixExponent}
    \exp(\bm{A} \tau) = \sum_{n = 0}^{\infty} \frac{1}{n!} \bm{A}^{n} \tau^{n}
\end{equation}

Práca s nekonečnými radmi je ale nepraktická. Vďaka Cayleymu-Hamiltonovmu teorému ale môžeme všetky vyššie mocniny ako $n$ matice $\bm{A}$ zapísať ako lineárnu kombináciu nižších mocnín:
\begin{equation}
    \bm{A}^{n} = \sum_{i = 0}^{n - 1} \alpha_{i} \bm{A}^{i}
\end{equation}
Maticový exponent (\ref{Controllability.Continuous.Equation:MatrixExponent}) môžeme teda prepísať do konečného radu, ktorého koeficienty sú závislé od času $\tau$:
\begin{equation}
    \exp(\bm{A} \tau) = \sum_{n = 0}^{n - 1} \alpha_i(\tau) \bm{A}^{i}
\end{equation}
kde $n$ v~tomto prípade označuje rád systému (počet prvkov stavového vektora $\bm{x}$). Rovnica (\ref{Controllability.Continuous.Equation:StateSpaceModelSolutionModified}) potom prejde do tvaru:
\begin{equation}
    \exp(-\bm{A} t_f) \bm{x}(t_f) - \exp(-\bm{A} t_i) \bm{x}(t_i) = \sum_{n = 0}^{n - 1} \left( \bm{A}^{i} \bm{B} \int_{t_i}^{t_f} \alpha_i(\tau) \bm{u}(\tau) d\tau \right)
\end{equation}
alebo miesto sumy môžeme použiť maticové násobenie:
\begin{multline}
    \exp(-\bm{A} t_f) \bm{x}(t_f) - \exp(-\bm{A} t_i) \bm{x}(t_i) \\ = 
    \underbrace{
        \begin{bmatrix}
            \bm{B} & \bm{A}\bm{B} & \cdots & \bm{A}^{n - 1} \bm{B}
        \end{bmatrix}
    }_{\bs{\mathcal{C}}}
    \begin{bmatrix}
        \int_{t_i}^{t_f} \alpha_0(\tau) \bm{u}(\tau) d\tau       \\
        \int_{t_i}^{t_f} \alpha_1(\tau) \bm{u}(\tau) d\tau       \\
        \vdots                                                 \\
        \int_{t_i}^{t_f} \alpha_{n - 1}(\tau) \bm{u}(\tau) d\tau \\
    \end{bmatrix}
\end{multline}
Získali sme sústavu rovníc, ktorej riešenie vráti postupnosť vstupov na dosiahnutie stavu $\bm{x}(t_f)$. Riešenie ale nie je triviálne, je ale postačujúce len poznať, že takéto riešenie existuje. Riešenie bude existovať ak matica riaditeľnosti $\bs{\mathcal{C}}$, s~rozmermi $n \times mn$, kde m je rozmer vstupného vektora $\bm{u}$ a $n$ je rozmer stavového vektora $\bm{x}$, má plnú hodnosť.

\begin{theorem}
    Spojitý lineárny systém (\ref{Controllability.Continuous.Equation:StateSpaceModel}) je riaditeľný pokiaľ matica riaditeľnosti $\bs{\mathcal{C}}$ má plnú hodnosť:
    \begin{equation}
        \label{Controllability.Continuous.Equation:ControllabilityTheorem}
        \rank \left\{ \bs{\mathcal{C}} \right\} = 
        \rank \left\{
            \begin{bmatrix}
                \bm{B} & \bm{A} \bm{B} & \cdots & \bm{A}^{n - 1} \bm{B}
            \end{bmatrix}
        \right\} = 
        n
    \end{equation}
\end{theorem}



\end{document}
