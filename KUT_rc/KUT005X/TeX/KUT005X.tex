\documentclass[a4paper, 10pt, ]{article}

\usepackage[slovak]{babel}

% ------------------------------

\usepackage[utf8]{inputenc}
\usepackage[T1]{fontenc}

\usepackage[left=4cm,
            right=4cm,
            top=2.1cm,
            bottom=2.6cm,
            footskip=7.5mm,
            twoside,
            marginparwidth=3.0cm,
            %showframe,
            ]{geometry}

\usepackage{graphicx}
\usepackage[dvipsnames]{xcolor}
% https://en.wikibooks.org/wiki/LaTeX/Colors

% ------------------------------

\usepackage{lmodern}

\usepackage[tt={oldstyle=false,proportional=true,monowidth}]{cfr-lm}
% https://mirror.szerverem.hu/ctan/fonts/cfr-lm/doc/cfr-lm.pdf

% ------------------------------

\usepackage{amsmath}
\usepackage{amssymb}
\usepackage{amsthm}

\usepackage{booktabs}
\usepackage{multirow}
\usepackage{array}
\usepackage{dcolumn}

\usepackage{natbib}

% ------------------------------

\hyphenpenalty=6000
\tolerance=1000

\def\naT{\mathsf{T}}

% ------------------------------

\makeatletter

    \def\@seccntformat#1{\protect\makebox[0pt][r]{\csname the#1\endcsname\hspace{4mm}}}

    \def\cleardoublepage{\clearpage\if@twoside \ifodd\c@page\else
    \hbox{}
    \vspace*{\fill}
    \begin{center}
    \phantom{}
    \end{center}
    \vspace{\fill}
    \thispagestyle{empty}
    \newpage
    \if@twocolumn\hbox{}\newpage\fi\fi\fi}

    \newcommand\figcaption{\def\@captype{figure}\caption}
    \newcommand\tabcaption{\def\@captype{table}\caption}

\makeatother

% ------------------------------

\usepackage{fancyhdr}
\fancypagestyle{plain}{%
\fancyhf{} % clear all header and footer fields
% \fancyfoot[C]{\sffamily {\bfseries \thepage}\ | {\scriptsize\oznacenieCasti}}
\fancyfoot[C]{\sffamily {\bfseries \thepage}{\color{Gray}\scriptsize$\,$z$\,$\pageref{LastPage}}\ | \includegraphics[height=5pt]{./COMMONFILES/KUT_logo_v0.1.pdf}{\scriptsize\KUTporadoveCislo}}
\renewcommand{\headrulewidth}{0pt}
\renewcommand{\footrulewidth}{0pt}}
\pagestyle{plain}

% ------------------------------

\usepackage{titlesec}
\titleformat{\paragraph}[hang]{\sffamily  \bfseries}{}{0pt}{}
\titlespacing*{\paragraph}{0mm}{3mm}{1mm}
\titlespacing*{\subparagraph}{0mm}{3mm}{1mm}

\titleformat*{\section}{\sffamily\Large\bfseries}
\titleformat*{\subsection}{\sffamily\large\bfseries}
\titleformat*{\subsubsection}{\sffamily\normalsize\bfseries}


% ------------------------------

\PassOptionsToPackage{hyphens}{url}
\usepackage[pdfauthor={},
            pdftitle={},
            pdfsubject={},
            pdfkeywords={},
            % hidelinks,
            colorlinks=false,
            breaklinks,
            ]{hyperref}


% ------------------------------

\graphicspath{%
{../fig_standalone/}%
{../../PY/fig/}%
{../../ML/fig/}%
{./fig/}%
}

% ------------------------------

\usepackage{enumitem}

\usepackage{lettrine}

% ------------------------------

\usepackage{lastpage}

\usepackage{microtype}

% ------------------------------

\usepackage{algorithm}
\usepackage[noend]{algpseudocode}
\makeatletter
\renewcommand{\ALG@name}{Algoritmus}
\makeatother
\usepackage{amsmath}
\usepackage{bbold}
\usepackage{calc}
\usepackage{dsfont}
\usepackage{mathtools}
\usepackage{tabto}


\newcommand{\mr}[1]{\mathrm{#1}}
\newcommand{\bs}[1]{\boldsymbol{#1}}
\newcommand{\bm}[1]{\mathbf{#1}}

\newcommand{\diff}[2]{\frac{\Delta #1}{\Delta #2}}
\newcommand{\der}[2]{\frac{d #1}{d #2}}
\newcommand{\parder}[2]{\frac{\partial #1}{\partial #2}}

\newcommand{\argmax}[0]{\mr{argmax}}
\newcommand{\diag}[0]{\mr{diag}}
\newcommand{\rank}[0]{\mr{rank}}
\newcommand{\trace}[0]{\mr{tr}}

\renewcommand{\Re}{\mr{Re}}
\renewcommand{\Im}{\mr{Im}}


\theoremstyle{definition}
\newtheorem{definition}{Definícia}[section]
\newtheorem{theorem}{Veta}[section]
\newtheorem{lemma}[theorem]{Lemma}
\newtheorem{example}{Príklad}[section]
\renewcommand*{\proofname}{Dôkaz}

% ------------------------------


% -----------------------------------------------------------------------------

\def\oznacenieCelku{Kolekcia učebných textov}

% -----------------------------------------------------------------------------


\def\KUTporadoveCislo{00X}

\def\oznacenieVerzie{v0.9}
% \def\oznacenieVerzie{\phantom{v1.0}}

\def\mesiacRok{apríl 2024}

\def\authorslabel{RM}





% -----------------------------------------------------------------------------

\begin{document}

% -----------------------------------------------------------------------------
% Uvodny nadpis

\noindent
\parbox[t][18mm][c]{0.3\textwidth}{%
\raisebox{-0.9\height}{%
\phantom{.}\includegraphics[height=18mm]{./COMMONFILES/URKFEIlogo.pdf}%
}%
}%
\parbox[t][18mm][c]{0.7\textwidth}{%
\raggedleft

\sffamily
\fontsize{16pt}{18pt}
\fontseries{sbc}
\selectfont

\noindent
\textcolor[rgb]{0.75, 0.75, 0.75}{\textls[25]{\oznacenieCelku}}
}%

\noindent
\parbox[t][16mm][b]{0.5\textwidth}{%
\raggedright

\color{Gray}
\sffamily

\fontsize{12pt}{12pt}
\selectfont
\mesiacRok

\fontsize{6pt}{10pt}
\selectfont
github.com/PracovnyBod/KUT

\fontsize{8pt}{10pt}
\selectfont
\authorslabel




}%
\parbox[t][16mm][b]{0.5\textwidth}{%
\raggedleft

\sffamily

\fontsize{6pt}{6pt}
\selectfont

\textcolor[rgb]{0.68, 0.68, 0.68}{\oznacenieVerzie}


\fontsize{14pt}{14pt}
\selectfont

\bfseries

\includegraphics[height=12pt]{./COMMONFILES/KUT_logo_v0.1.pdf}%
{%
\textls[-50]{\KUTporadoveCislo}
}%
}%

% -----------------------------------------------------------------------------




\vspace{6mm}

% ---------------------------------------------
\sffamily
\bfseries
\fontsize{18pt}{21pt}
\selectfont

\begin{flushleft}
    O algebraických štruktúrach s dvomi binárnymi operáciami
\end{flushleft}

\bigskip

% -----------------------------------------------------------------------------
\normalsize
\normalfont
% -----------------------------------------------------------------------------



\section{Algebraické štruktúry s dvomi binárnymi operáciami}
\label{TwoBinaryOperationsStructures}

Neskúsenejší čitateľ pravdepodobne po prečítaní pojmu lineárna algebra už radšej ďalej nepokračoval. Skúsenejší čitateľ sa už možno aj stretol s pojmami ako pole, teleso alebo okruh, avšak neprišlo mu jednoznačné prečo sú dané veci tak pomenované a na čo vlastne slúžia. Všetko sú to predsa pojmy z~abstraktnej matematiky, názvy teda niekedy treba brať s~nadhľadom.

Gro abstraktnej matematiky spočíva v triedení - škatuľkovaní podľa určitých vlastností. Stanovíme si nejaké spoločné vlastnosti, ktoré sú spoločné pre skupinu objektov, tieto vlastnosti sa nazývajú axiómy. Pokiaľ teda niečo spĺňa všetky axiómy z nejakej skupiny, tak môžeme tvrdiť, že do tejto skupiny aj patrí - zaškatuľkujeme to. Z~týchto axiómov sú potom odvoditeľné lemmy a vety, ktoré sú používané v~iných odvetviach a~vieme, že pre všetky naše zaškatuľkované veci platia a sú teda použiteľné.

Príklad lineárna kombinácia sa používa vo frekvenčnej analýze a~pri riešení lineárnych diferenciálnych rovníc, čiže v~podstate sa s~ňou stretávame neustále v~lineárnej teórii riadenia. Ďalším príkladom, môže byť stavová rovina, v~ktorej stavy sú vektory, na to, aby sme vedeli povedať čo je to vektor a~aké vlastnosti má potrebujeme definíciu vektorového priestoru.

V tejto časti si uvedieme štruktúry, ktoré majú dve binárne operácie: operáciu súčtu $\oplus$ a~operáciiu násobenia $\odot$. Symboly pre operácie sú zámerne uvedené v krúžku, aby sme ich odlíšili od štandardne definovaného $+$ a $\cdot$, pretože dané operácie môžu byť definované podľa čitateľovej fantázie. Rovnako rozlišujeme špeciálnu nulu $\mathbb{0}$ a jednotku $\mathbb{1}$, ktoré tak isto nemusia zodpovedať svojím numerickým náprotivkom.

\begin{definition}
    \label{TwoBinaryOperationsStructures.Definition:TwoBinaryOperationsStructures}
    Axiómy algebraických štruktúr s dvomi binárnymi operáciami $\oplus$ a $\odot$:
    \begin{enumerate}\addtocounter{enumi}{-1}
        \TabPositions{1cm}
        \item Definujeme dve binárne operácie súčet $\oplus$ a násobenie $\odot$, na objekte $\mathbb{P}$ s prvkami $a, b$: \\
        \tab\tab $\oplus: \ \mathbb{P}\times\mathbb{P} \owns \{a, b\} \ \rightarrow \ a \oplus b \in \mathbb{P}$ \tab\tab $\odot: \ \mathbb{P}\times\mathbb{P} \owns \{a, b\} \ \rightarrow \ a \odot b \in \mathbb{P}$

        Z tohto axiómu vidíme, že pokiaľ vykonáme operáciu $\oplus$ alebo $\odot$, tak výsledok je stále z objektu $\mathbb{P}$.
        
        \item Operácia $\oplus$ je asociatívna: \\
        \tab\tab $(a \oplus b) \oplus c = a \oplus (b \oplus c)$
        
        \item Existuje prvok identity $\mathbb{0}$ ("nulový prvok") pre operáciu $\oplus$: \\
        \tab\tab $\exists \ \mathbb{0} \in \mathbb{P}: \ \mathbb{0} \oplus a = a \oplus \mathbb{0} = a$
        
        \item Pre každý prvok $a$ z $\mathbb{P}$ pre operáciu $\oplus$ xistuje inverzný prvok $-a$: \\
        \tab\tab $\forall \ a \in \mathbb{P} \ \exists \ -a: \ a \oplus (-a) = \mathbb{0}$
        
        \item Operácia $\oplus$ je komutatívna: \\
        \tab\tab $a \oplus b = b \oplus a$
        
        \item Distributívnosť operácie $\odot$ zľava: \\
        \tab\tab $a \odot (b \oplus c) = (a \odot b) \oplus (a \odot c)$
        
        \item Distributívnosť operácie $\odot$ sprava: \\
        \tab\tab $(b \oplus c) \odot a = (b \odot a) \oplus (c \odot a)$
        
        \item Operácia $\odot$ je asociatívna: \\
        \tab\tab $(a \odot b) \odot c = a \odot (b \odot c)$

        \item Existuje prvok identity $\mathbb{1}$ pre operáciu $\oplus$: \\
        \tab\tab $\exists \ \mathbb{1} \in \mathbb{P}: \ \mathbb{1} \odot a = a \odot \mathbb{1} = a$

        \item Pre každý prvok $a$ z $\mathbb{P}$ pre operáciu $\odot$ existuje inverzný prvok $a^{-1}$: \\
        \tab\tab $\forall a \neq 0 \ \exists \ a^{-1}: \ a \odot a^{-1} = a^{-1} \odot a = \mathbb{1}$

        \item Operácia $\odot$ je komutatívna: \\
        \tab\tab $a \odot b = b \odot a$
    \end{enumerate}
\end{definition}

Pokiaľ sú splnené axiómy (0)-(10) hovoríme o poli, pre (0)-(9) hovoríme o telese, pre (0)-(7) hovoríme o okruhu, pokiaľ pre okruh platí aj axióm (8), hovoríme o okruhu s~jednotkou. Príkladom poľa sú napríklad reálne a komplexné čísla, nechávame na čitateľovom zamyslení si dané tvrdenie overiť kontrolovou jednotlivých axiómov, ďalším zamyslením môže byť či aj celé čísla sú poľom. Nad poľom definujeme ďalší dôležitý pojem z lineárnej algebry - vektorový priestor.



\section{Vektorový priestor}
\label{VectorSpace}

\begin{definition}[Vektorový priestor]
    \label{VectorSpace.Definition:VectorSpace}
    Vektorový priestor $V$ nad poľom $K$ je neprázdna množina s dvomi definovanými operáciami $\oplus: V \times V \rightarrow V$  a $\odot: K \times V \rightarrow V$, ktorá spĺňa nasledujúcich 8 axiómov:
    \begin{enumerate}
        \TabPositions{1cm}
        \item Asociatívnosť: \\
        \tab\tab $\bm{u} \oplus (\bm{v} \oplus \bm{w}) = (\bm{u} \oplus \bm{v}) \oplus \bm{w}, \quad \bm{u}, \bm{v}, \bm{w} \in V$
        
        \item Komutatívnosť: \\
        \tab\tab $\bm{u} \oplus \bm{v} = \bm{v} \oplus \bm{u}, \quad \bm{u}, \bm{v} \in V$
        
        \item Prvok identity vektorového súčtu: \\
        \tab\tab $\bm{u} \oplus \bm{0} = \bm{u}, \quad \bm{u}, \bm{0} \in V$
        
        \item Inverzný prvok: \\
        \tab\tab $\bm{u} \oplus \bm{v} = \bm{0}, \quad \bm{u}, \bm{v}, \bm{0} \in V$
        
        \item Kompatibilita skalárneho násobenia s násobením poľa \\
        \tab\tab $\alpha \odot (\beta \odot \bm{u}) = (\alpha \beta) \odot \bm{u}, \quad \alpha, \beta \in K, \quad \bm{u} \in V$
        
        \item Prvok identity skalárneho násobenia: \\
        \tab\tab $\mathbb{1} \odot \bm{u} = \bm{u}, \quad \mathbb{1} \in K, \quad \bm{u} \in V$
        
        \item Distributívnosť vzhľadom na vektorový súčet: \\
        \tab\tab $\alpha \odot (\bm{u} \oplus \bm{v}) = \alpha \odot \bm{u} \oplus \alpha \odot \bm{v}, \quad \alpha \in K, \quad \bm{u}, \bm{v} \in V$
        
        \item Distributívnosť vzhľadom na súčet poľa: \\
        \tab\tab $(\alpha + \beta) \odot \bm{u} = \alpha \odot \bm{u} \oplus \beta \odot \bm{u}, \quad \alpha, \beta \in K, \quad \bm{u} \in V$
    \end{enumerate}
\end{definition}

V definícii (\ref{VectorSpace.Definition:VectorSpace}) sme zámerne označili operácie $+$ a $\cdot$ v krúžku, aby sme ich rozlíšili od štandardne definovaného súčtu a súčinu, je možné ale používať aj označenie bez krúžku. Prvky $\bm{0}$ a $\mathbb{1}$ sú špeciálne a pre každý vektorový priestor originálne, nemusia nutne zodpovedať číselnej $0$ a $1$.

\begin{example}
    Vyšetrite či komplexné čísla $\bm{z} = \alpha + i \beta$ so štandardne definovanými operáciami $+, \cdot$ tvoria vektorový priestor $V$ nad reálnymi číslami $\mathbb{R}$, $\bm{z} \in V$, $\alpha, \beta \in \mathbb{R}$.

    \begin{enumerate}
        \item
        \begin{align*}
            \alpha_1 + i \beta_1 + (\alpha_2 + i \beta_2 + \alpha_3 + i \beta_3) &= 
            \alpha_1 + i \beta_1 + (\alpha_2 + \alpha_3 + i (\beta_2 + \beta_3)) \\ &= 
            \alpha_1 + \alpha_2 + \alpha_3 + i (\beta_1 + \beta_2 + \beta_3) \\ 
            (\alpha_1 + i \beta_1 + \alpha_2 + i \beta_2) + \alpha_3 + i \beta_3 &= 
            (\alpha_1 + \alpha_2 i (\beta_1 + \beta_2)) + \alpha_3 + i\beta_3 \\ &= 
            \alpha_1 + \alpha_2 + \alpha_3 + i (\beta_1 + \beta_2 + \beta_3) 
        \end{align*}
        
        \item
        \begin{align*}
            \alpha_1 + i \beta_1 + \alpha_2 + i \beta_2 &= 
            \alpha_1 + \alpha_2 + i (\beta_1 + \beta_2) \\ 
            \alpha_2 + i \beta_2 + \alpha_1 + i \beta_1&= 
            \alpha_2 + \alpha_1 + i (\beta_2 + \beta_1) \\ &= 
            \alpha_1 + \alpha_2 + i (\beta_1 + \beta_2) 
        \end{align*}

        \item
        \begin{align*}
            &\alpha + i \beta + o_r + i o_i = 
            \alpha + i \beta \\ 
            &\alpha + o_r = \alpha \ \Longrightarrow \ o_r = 0 \\
            &\beta + o_i = \beta \ \Longrightarrow \ o_i = 0 \\
            & \bm{0} = 0 + i0 
        \end{align*}

        \item
        \begin{align*}
            &\alpha_1 + i \beta_1 + \alpha_2 + i\beta_2 = 0 + i0 \\
            &\alpha_1 + \alpha_2 = 0 \ \Longrightarrow \ \alpha_1 = -\alpha_2 \\
            &\beta_1 + \beta_2 = 0 \ \Longrightarrow \ \beta_1 = -\beta_2 \\
            & \bm{z} + (-\bm{z}) = \bm{0} 
        \end{align*}

        \item $\mu, \ \nu \in \mathbb{R}$
        \begin{align*}
            \mu (\nu (\alpha + i \beta)) &= \mu (\nu \alpha + i \nu \beta)) = \mu \nu \alpha + i \mu \nu \beta \\
            (\mu \nu) (\alpha + i \beta) &= \mu \nu \alpha + i \mu \nu \beta \\
        \end{align*}

        \item
        \begin{align*}
            &\mathbb{1} \alpha + i \mathbb{1} \beta = \alpha + i \beta \\
            &\mathbb{1} \alpha = \alpha \ \Longrightarrow \ \mathbb{1} = 1 \\
            &\mathbb{1} \beta = \beta \ \Longrightarrow \ \mathbb{1} = 1 \\
            &\mathbb{1} = 1 
        \end{align*}

        \item $\mu, \ \nu \in \mathbb{R}$
        \begin{align*}
            \mu (\alpha_1 + i \beta_1 + \alpha_2 + i\beta_2) &= 
            \mu (\alpha_1 + \alpha_2 + i (\beta_1 + \beta_2)) \\ &= 
            \mu(\alpha_1 + \alpha_2) + i \mu (\beta_1 + \beta_2) \\ & = 
            \mu \alpha_1 + \mu \alpha_2 + i (\mu \beta_1 + \mu \beta_2) \\ 
            \mu (\alpha_1 + i \beta_1) + \mu (\alpha_2 + i \beta_2) &= 
            \mu \alpha_1 + i \mu \beta_1 + \mu \alpha_2 + i \mu \beta_2 \\ &=
            \mu \alpha_1 + \mu \alpha_2 + i (\mu \beta_1 + \mu \beta_2)  
        \end{align*}

        \item $\mu, \ \nu \in \mathbb{R}$
        \begin{align*}
            (\mu + \nu) (\alpha + i \beta) &= 
            (\mu + \nu) \alpha + i (\mu + \nu) \beta \\ &=
            \mu \alpha + \nu \alpha + i (\mu \beta + \nu \beta) \\
            \mu (\alpha + i \beta) + \nu (\alpha + i \beta) &= 
            \mu \alpha + i \mu \beta + \nu \alpha + i \nu \beta \\ &=
            \mu \alpha + \nu \alpha + i (\mu \beta + \nu \beta)  
        \end{align*}
    \end{enumerate}
    Vidíme, že všetky axiómy sú splnené a môžeme teda tvrdiť, že komplexné čísla tvoria vektorový priestor nad reálnymi číslami.
\end{example}



\end{document}
