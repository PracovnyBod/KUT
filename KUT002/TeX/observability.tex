\documentclass[a4paper, 10pt, ]{article}

\usepackage[slovak]{babel}

% ------------------------------

\usepackage[utf8]{inputenc}
\usepackage[T1]{fontenc}

\usepackage[left=4cm,
            right=4cm,
            top=2.1cm,
            bottom=2.6cm,
            footskip=7.5mm,
            twoside,
            marginparwidth=3.0cm,
            %showframe,
            ]{geometry}

\usepackage{graphicx}
\usepackage[dvipsnames]{xcolor}
% https://en.wikibooks.org/wiki/LaTeX/Colors

% ------------------------------

\usepackage{lmodern}

\usepackage[tt={oldstyle=false,proportional=true,monowidth}]{cfr-lm}
% https://mirror.szerverem.hu/ctan/fonts/cfr-lm/doc/cfr-lm.pdf

% ------------------------------

\usepackage{amsmath}
\usepackage{amssymb}
\usepackage{amsthm}

\usepackage{booktabs}
\usepackage{multirow}
\usepackage{array}
\usepackage{dcolumn}

\usepackage{natbib}

% ------------------------------

\hyphenpenalty=6000
\tolerance=1000

\def\naT{\mathsf{T}}

% ------------------------------

\makeatletter

    \def\@seccntformat#1{\protect\makebox[0pt][r]{\csname the#1\endcsname\hspace{4mm}}}

    \def\cleardoublepage{\clearpage\if@twoside \ifodd\c@page\else
    \hbox{}
    \vspace*{\fill}
    \begin{center}
    \phantom{}
    \end{center}
    \vspace{\fill}
    \thispagestyle{empty}
    \newpage
    \if@twocolumn\hbox{}\newpage\fi\fi\fi}

    \newcommand\figcaption{\def\@captype{figure}\caption}
    \newcommand\tabcaption{\def\@captype{table}\caption}

\makeatother

% ------------------------------

\usepackage{fancyhdr}
\fancypagestyle{plain}{%
\fancyhf{} % clear all header and footer fields
% \fancyfoot[C]{\sffamily {\bfseries \thepage}\ | {\scriptsize\oznacenieCasti}}
\fancyfoot[C]{\sffamily {\bfseries \thepage}{\color{Gray}\scriptsize$\,$z$\,$\pageref{LastPage}}\ | \includegraphics[height=5pt]{./COMMONFILES/KUT_logo_v0.1.pdf}{\scriptsize\KUTporadoveCislo}}
\renewcommand{\headrulewidth}{0pt}
\renewcommand{\footrulewidth}{0pt}}
\pagestyle{plain}

% ------------------------------

\usepackage{titlesec}
\titleformat{\paragraph}[hang]{\sffamily  \bfseries}{}{0pt}{}
\titlespacing*{\paragraph}{0mm}{3mm}{1mm}
\titlespacing*{\subparagraph}{0mm}{3mm}{1mm}

\titleformat*{\section}{\sffamily\Large\bfseries}
\titleformat*{\subsection}{\sffamily\large\bfseries}
\titleformat*{\subsubsection}{\sffamily\normalsize\bfseries}


% ------------------------------

\PassOptionsToPackage{hyphens}{url}
\usepackage[pdfauthor={},
            pdftitle={},
            pdfsubject={},
            pdfkeywords={},
            % hidelinks,
            colorlinks=false,
            breaklinks,
            ]{hyperref}


% ------------------------------

\graphicspath{%
{../fig_standalone/}%
{../../PY/fig/}%
{../../ML/fig/}%
{./fig/}%
}

% ------------------------------

\usepackage{enumitem}

\usepackage{lettrine}

% ------------------------------

\usepackage{lastpage}

\usepackage{microtype}

% ------------------------------

\usepackage{algorithm}
\usepackage[noend]{algpseudocode}
\makeatletter
\renewcommand{\ALG@name}{Algoritmus}
\makeatother
\usepackage{amsmath}
\usepackage{bbold}
\usepackage{calc}
\usepackage{dsfont}
\usepackage{mathtools}
\usepackage{tabto}


\newcommand{\mr}[1]{\mathrm{#1}}
\newcommand{\bs}[1]{\boldsymbol{#1}}
\newcommand{\bm}[1]{\mathbf{#1}}

\newcommand{\diff}[2]{\frac{\Delta #1}{\Delta #2}}
\newcommand{\der}[2]{\frac{d #1}{d #2}}
\newcommand{\parder}[2]{\frac{\partial #1}{\partial #2}}

\newcommand{\argmax}[0]{\mr{argmax}}
\newcommand{\diag}[0]{\mr{diag}}
\newcommand{\rank}[0]{\mr{rank}}
\newcommand{\trace}[0]{\mr{tr}}

\renewcommand{\Re}{\mr{Re}}
\renewcommand{\Im}{\mr{Im}}


\theoremstyle{definition}
\newtheorem{definition}{Definícia}[section]
\newtheorem{theorem}{Veta}[section]
\newtheorem{lemma}[theorem]{Lemma}
\newtheorem{example}{Príklad}[section]
\renewcommand*{\proofname}{Dôkaz}

% ------------------------------


% -----------------------------------------------------------------------------

\def\oznacenieCelku{Kolekcia učebných textov}

% -----------------------------------------------------------------------------

\def\KUTporadoveCislo{002}

% \def\oznacenieVerzie{v1.0}
\def\oznacenieVerzie{\phantom{v1.0}}

\def\mesiacRok{február 2024}

\def\authorslabel{}






% -----------------------------------------------------------------------------

\begin{document}


% -----------------------------------------------------------------------------
% Uvodny nadpis

\noindent
\parbox[t][18mm][c]{0.3\textwidth}{%
\raisebox{-0.9\height}{%
\phantom{.}\includegraphics[height=18mm]{../../COMMONFILES/URKFEIlogo.pdf}%
}%
}%
\parbox[t][18mm][c]{0.7\textwidth}{%
\raggedleft



\sffamily
\fontsize{16pt}{18pt}
\fontseries{sbc}
\selectfont

\noindent
\textcolor[rgb]{0.75, 0.75, 0.75}{\textls[30]{\oznacenieCelku}}
}%

\noindent
\parbox[t][16mm][b]{0.5\textwidth}{%
\raggedright

\color{Gray}
\sffamily


\fontsize{8pt}{10pt}
\selectfont

\authorslabel

\fontsize{6pt}{10pt}
\selectfont

urk.fei.stuba.sk

\fontsize{12pt}{12pt}
\selectfont
\mesiacRok


}%
\parbox[t][16mm][b]{0.5\textwidth}{%
\raggedleft

\sffamily
% \color{Gray}
% \lstyle 

\fontsize{6pt}{6pt}
\selectfont

\textcolor[rgb]{0.68, 0.68, 0.68}{\oznacenieVerzie}

% \textcolor[rgb]{0.68, 0.68, 0.68}{počet strán: \pageref{LastPage}}

\fontsize{14pt}{14pt}
% \fontseries{sbc}
\selectfont

\bfseries

\includegraphics[height=12pt]{../../COMMONFILES/KUT_logo_v0.1.pdf}%
% \oznacenieCasti%
{%
% \lstyle
% \pstyle
\textls[-50]{\KUTporadoveCislo}
}%

}%

% -----------------------------------------------------------------------------




\vspace{6mm}

% ---------------------------------------------
\sffamily
\bfseries
\fontsize{18pt}{21pt}
\selectfont

\begin{flushleft}
    O základných vlastnostiach lineárnych systémov
\end{flushleft}

\bigskip

% -----------------------------------------------------------------------------
\normalsize
\normalfont
% -----------------------------------------------------------------------------



\section{Pozorovateľnosť}


\subsection{Diskrétny systém}

\begin{equation}
    \label{Observability.Discrete.Equation:StateSpaceModel}
    \bm{x}_{n + 1} = \bm{A} \bm{x}_{n}
\end{equation}

\begin{equation}
    \bm{y}_n = \bm{C} \bm{x}_{n}
\end{equation}

\begin{subequations}
    \begin{align}
        \bm{y}_0       &= \bm{C} \bm{x}_{0}                \\
        \bm{y}_1       &= \bm{C} \bm{A} \bm{x}_{0}         \\
        &\vdotswithin{=}                                   \\
        \bm{y}_{n - 1} &= \bm{C} \bm{A}^{n - 1} \bm{x}_{0}
    \end{align}
\end{subequations}

\begin{equation}
    \begin{bmatrix}
        \bm{y}_0       \\
        \bm{y}_1       \\
        \vdots         \\
        \bm{y}_{n - 1} \\
    \end{bmatrix} = 
    \underbrace{
        \begin{bmatrix}
            \bm{C}                \\
            \bm{C} \bm{A}         \\
            \vdots                \\
            \bm{C} \bm{A}^{n - 1} \\
        \end{bmatrix}
    }_{\bs{\mathcal{O}}}
    \bm{x}_0
\end{equation}

\begin{theorem}
    Diskrétny lineárny systém (\ref{Observability.Discrete.Equation:StateSpaceModel}) je pozorovateľný pokiaľ matica pozorovateľnosti $\bs{\mathcal{O}}$ má plnú hodnosť:
    \begin{equation}
        \label{Observability.Discrete.Equation:ObservabilityTheorem}
        \rank \left\{ \bs{\mathcal{O}} \right\} = 
        \rank \left\{
            \begin{bmatrix}
                \bm{C}                \\
                \bm{C} \bm{A}         \\
                \vdots                \\
                \bm{C} \bm{A}^{n - 1} \\
            \end{bmatrix}
        \right\} = 
        n
    \end{equation}
\end{theorem}


\subsection{Spojitý systém}

\begin{equation}
    \label{Observability.Continuous.Equation:StateSpaceModel}
    \dot{\bm{x}} = \bm{A} \bm{x}
\end{equation}

\begin{equation}
    \bm{y} = \bm{C} \bm{x}
\end{equation}

\begin{subequations}
    \begin{align}
        \dot{\bm{y}}(t_i)     &= \bm{C} \bm{x}_{0}                \\
        \ddot{\bm{y}}(t_i)    &= \bm{C} \bm{A} \bm{x}_{0}         \\
        &\vdotswithin{=}                                          \\
        \bm{y}^{(n - 1)}(t_i) &= \bm{C} \bm{A}^{n - 1} \bm{x}_{0}
    \end{align}
\end{subequations}

\begin{equation}
    \begin{bmatrix}
        \dot{\bm{y}}(t_i)     \\
        \ddot{\bm{y}}(t_i)    \\
        \vdots                \\
        \bm{y}^{(n - 1)}(t_i)
    \end{bmatrix} = 
    \underbrace{
        \begin{bmatrix}
            \bm{C}                \\
            \bm{C} \bm{A}         \\
            \vdots                \\
            \bm{C} \bm{A}^{n - 1} \\
        \end{bmatrix}
    }_{\bs{\mathcal{O}}}
    \bm{x}(t_i)
\end{equation}

\begin{theorem}
    Spojitý lineárny systém (\ref{Observability.Continuous.Equation:StateSpaceModel}) je pozorovateľný pokiaľ matica pozorovateľnosti $\bs{\mathcal{O}}$ má plnú hodnosť:
    \begin{equation}
        \label{Observability.Continuous.Equation:ObservabilityTheorem}
        \rank \left\{ \bs{\mathcal{O}} \right\} = 
        \rank \left\{
            \begin{bmatrix}
                \bm{C}                \\
                \bm{C} \bm{A}         \\
                \vdots                \\
                \bm{C} \bm{A}^{n - 1} \\
            \end{bmatrix}
        \right\} = 
        n
    \end{equation}
\end{theorem}



\end{document}
