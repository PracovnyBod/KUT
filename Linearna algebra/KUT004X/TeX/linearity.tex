\documentclass[a4paper, 10pt, ]{article}

\usepackage[slovak]{babel}

% ------------------------------

\usepackage[utf8]{inputenc}
\usepackage[T1]{fontenc}

\usepackage[left=4cm,
            right=4cm,
            top=2.1cm,
            bottom=2.6cm,
            footskip=7.5mm,
            twoside,
            marginparwidth=3.0cm,
            %showframe,
            ]{geometry}

\usepackage{graphicx}
\usepackage[dvipsnames]{xcolor}
% https://en.wikibooks.org/wiki/LaTeX/Colors

% ------------------------------

\usepackage{lmodern}

\usepackage[tt={oldstyle=false,proportional=true,monowidth}]{cfr-lm}
% https://mirror.szerverem.hu/ctan/fonts/cfr-lm/doc/cfr-lm.pdf

% ------------------------------

\usepackage{amsmath}
\usepackage{amssymb}
\usepackage{amsthm}

\usepackage{booktabs}
\usepackage{multirow}
\usepackage{array}
\usepackage{dcolumn}

\usepackage{natbib}

% ------------------------------

\hyphenpenalty=6000
\tolerance=1000

\def\naT{\mathsf{T}}

% ------------------------------

\makeatletter

    \def\@seccntformat#1{\protect\makebox[0pt][r]{\csname the#1\endcsname\hspace{4mm}}}

    \def\cleardoublepage{\clearpage\if@twoside \ifodd\c@page\else
    \hbox{}
    \vspace*{\fill}
    \begin{center}
    \phantom{}
    \end{center}
    \vspace{\fill}
    \thispagestyle{empty}
    \newpage
    \if@twocolumn\hbox{}\newpage\fi\fi\fi}

    \newcommand\figcaption{\def\@captype{figure}\caption}
    \newcommand\tabcaption{\def\@captype{table}\caption}

\makeatother

% ------------------------------

\usepackage{fancyhdr}
\fancypagestyle{plain}{%
\fancyhf{} % clear all header and footer fields
% \fancyfoot[C]{\sffamily {\bfseries \thepage}\ | {\scriptsize\oznacenieCasti}}
\fancyfoot[C]{\sffamily {\bfseries \thepage}{\color{Gray}\scriptsize$\,$z$\,$\pageref{LastPage}}\ | \includegraphics[height=5pt]{./COMMONFILES/KUT_logo_v0.1.pdf}{\scriptsize\KUTporadoveCislo}}
\renewcommand{\headrulewidth}{0pt}
\renewcommand{\footrulewidth}{0pt}}
\pagestyle{plain}

% ------------------------------

\usepackage{titlesec}
\titleformat{\paragraph}[hang]{\sffamily  \bfseries}{}{0pt}{}
\titlespacing*{\paragraph}{0mm}{3mm}{1mm}
\titlespacing*{\subparagraph}{0mm}{3mm}{1mm}

\titleformat*{\section}{\sffamily\Large\bfseries}
\titleformat*{\subsection}{\sffamily\large\bfseries}
\titleformat*{\subsubsection}{\sffamily\normalsize\bfseries}


% ------------------------------

\PassOptionsToPackage{hyphens}{url}
\usepackage[pdfauthor={},
            pdftitle={},
            pdfsubject={},
            pdfkeywords={},
            % hidelinks,
            colorlinks=false,
            breaklinks,
            ]{hyperref}


% ------------------------------

\graphicspath{%
{../fig_standalone/}%
{../../PY/fig/}%
{../../ML/fig/}%
{./fig/}%
}

% ------------------------------

\usepackage{enumitem}

\usepackage{lettrine}

% ------------------------------

\usepackage{lastpage}

\usepackage{microtype}

% ------------------------------

\usepackage{algorithm}
\usepackage[noend]{algpseudocode}
\makeatletter
\renewcommand{\ALG@name}{Algoritmus}
\makeatother
\usepackage{amsmath}
\usepackage{bbold}
\usepackage{calc}
\usepackage{dsfont}
\usepackage{mathtools}
\usepackage{tabto}


\newcommand{\mr}[1]{\mathrm{#1}}
\newcommand{\bs}[1]{\boldsymbol{#1}}
\newcommand{\bm}[1]{\mathbf{#1}}

\newcommand{\diff}[2]{\frac{\Delta #1}{\Delta #2}}
\newcommand{\der}[2]{\frac{d #1}{d #2}}
\newcommand{\parder}[2]{\frac{\partial #1}{\partial #2}}

\newcommand{\argmax}[0]{\mr{argmax}}
\newcommand{\diag}[0]{\mr{diag}}
\newcommand{\rank}[0]{\mr{rank}}
\newcommand{\trace}[0]{\mr{tr}}

\renewcommand{\Re}{\mr{Re}}
\renewcommand{\Im}{\mr{Im}}


\theoremstyle{definition}
\newtheorem{definition}{Definícia}[section]
\newtheorem{theorem}{Veta}[section]
\newtheorem{lemma}[theorem]{Lemma}
\newtheorem{example}{Príklad}[section]
\renewcommand*{\proofname}{Dôkaz}

% ------------------------------


% -----------------------------------------------------------------------------

\def\oznacenieCelku{Kolekcia učebných textov}

% -----------------------------------------------------------------------------

\def\KUTporadoveCislo{001}

% \def\oznacenieVerzie{v1.0}
\def\oznacenieVerzie{\phantom{v1.0}}

\def\mesiacRok{február 2024}

\def\authorslabel{}

% -----------------------------------------------------------------------------

% -----------------------------------------------------------------------------

\begin{document}


% -----------------------------------------------------------------------------
% Uvodny nadpis

\noindent
\parbox[t][18mm][c]{0.3\textwidth}{%
\raisebox{-0.9\height}{%
\phantom{.}\includegraphics[height=18mm]{./COMMONFILES/URKFEIlogo.pdf}%
}%
}%
\parbox[t][18mm][c]{0.7\textwidth}{%
\raggedleft



\sffamily
\fontsize{16pt}{18pt}
\fontseries{sbc}
\selectfont

\noindent
\textcolor[rgb]{0.75, 0.75, 0.75}{\textls[30]{\oznacenieCelku}}
}%

\noindent
\parbox[t][16mm][b]{0.5\textwidth}{%
\raggedright

\color{Gray}
\sffamily


\fontsize{8pt}{10pt}
\selectfont

\authorslabel

\fontsize{6pt}{10pt}
\selectfont

urk.fei.stuba.sk

\fontsize{12pt}{12pt}
\selectfont
\mesiacRok


}%
\parbox[t][16mm][b]{0.5\textwidth}{%
\raggedleft

\sffamily
% \color{Gray}
% \lstyle 

\fontsize{6pt}{6pt}
\selectfont

\textcolor[rgb]{0.68, 0.68, 0.68}{\oznacenieVerzie}

% \textcolor[rgb]{0.68, 0.68, 0.68}{počet strán: \pageref{LastPage}}

\fontsize{14pt}{14pt}
% \fontseries{sbc}
\selectfont

\bfseries

\includegraphics[height=12pt]{./COMMONFILES/KUT_logo_v0.1.pdf}%
% \oznacenieCasti%
{%
% \lstyle
% \pstyle
\textls[-50]{\KUTporadoveCislo}
}%

}%

% -----------------------------------------------------------------------------




\vspace{6mm}

% ---------------------------------------------
\sffamily
\bfseries
\fontsize{18pt}{21pt}
\selectfont

\begin{flushleft}
	O lineárnych zobrazeniach
\end{flushleft}

\bigskip

% -----------------------------------------------------------------------------
\normalsize
\normalfont
% -----------------------------------------------------------------------------




\section{Lineárne zobrazenie}
\label{Linearity}

Lineárnosť (lineárne zobrazenie) je matematická vlastnosť funkcií/systémov. Lineárne zobrazenie je transformácia $U \rightarrow V$, ktorá zobrazí vektory z vektorového priestoru $V$ do vektorového priestoru $U$, pričom zachováva operácie súčtu vektorov a~násobenia vektorom skalárom.

\begin{definition}(Lineárne zobrazenie)
    Nech $U$ a $V$ sú vektorové priestory nad poľom $K$ Hovoríme, že $f: V \rightarrow U$ je lineárne zobrazenie ak pre ľubovoľné vektory $\bm{x}, \bm{y} \in V$ a~ľubovoľný skalár $\alpha \in K$ platí:
    \begin{subequations}
        \begin{align}
            \label{Linearity.Equation:Aditivity}
            f(\bm{x} + \bm{y}) &= f(\bm{x}) + f(\bm{y}) \\[6pt]
            \label{Linearity.Equation:Homogeneity}
            f(\alpha \bm{x})   &= \alpha f(\bm{x})
        \end{align}
    \end{subequations}
    Vlastnosť (\ref{Linearity.Equation:Aditivity}) nazývame \textit{aditivitou} a vlastnosť (\ref{Linearity.Equation:Homogeneity}) nazývame \textit{homogenitou}.
\end{definition}

\begin{example}
    \begin{subequations}
        Uvedieme príklad a overíme, či funkcia:
        \begin{equation}
            \label{Linearity.Equation:Example1}
            f(x) = a x
        \end{equation}
        je lineárna.
        
        Najskôr overíme aditivitu, výpočet realizujeme pre ľavú a pravú stranu rovnice (\ref{Linearity.Equation:Aditivity}) zvlášť:
        \begin{align}
            f(x + y) &= a (x + y) = a x + a y \\[6pt]
            f(x) + f(y) & = a x + a y
        \end{align}
        Vidíme, že obidve strany rovnice sa rovnajú, aditivita je splnená. Ďalej overujeme homogenitu (\ref{Linearity.Equation:Homogeneity}), keďže $a$ a $\alpha$ sú čísla (reálne/komplexné), tak môžeme písať:
        \begin{align}
            f(\alpha x) &=  a (\alpha x) = \alpha (a x) \\[6pt]
            \alpha f(x) &= \alpha (a x)
        \end{align}
        Teda aj homogenita je splnená a funkcia (\ref{Linearity.Equation:Example1}) je lineárne zobrazenie.
    \end{subequations}
\end{example}

\begin{example}
    \begin{subequations}
        Uvedieme teraz príklad nelineárnej funkcie:
        \begin{equation}
            \label{Linearity.Equation:Example2}
            f(x) = x^2
        \end{equation}
        \textit{Aditivita:}
        \begin{align}
            f(x + y)    &= (x + y)^2 = x^2 + 2xy + y^2 \\[6pt]
            f(x) + f(y) & = x^2 + y^2
        \end{align}
        \textit{Homogenita:}
        \begin{align}
            f(\alpha x) &= (\alpha x)^2 = \alpha^2 x^2 \\[6pt]
            \alpha f(x) &= \alpha x^2
        \end{align}
        Ani jedna z vlastností nie je splnená, funkcia je teda nelineárna.
    \end{subequations}
\end{example}

\begin{example}
    \begin{subequations}
        Nakoniec ešte ukážeme príklad pre funkciu:
        \begin{equation}
            \label{Linearity.Equation:Example3}
            f(x) = a x + b
        \end{equation}
        Funkciu (\ref{Linearity.Equation:Example1}) sme rozšírili o konštantu $b$ a overíme teda či platí aditivita a homogenita.
        
        \vspace*{10pt}

        \noindent \textit{Aditivita:}
        \begin{align}
            f(x + y) &= a (x + y) + b = a x + a y + b \\[6pt]
            f(x) + f(y) & = a x + b + a y + b = a x + a y + 2b
        \end{align}
        \textit{Homogenita:}
        \begin{align}
            f(\alpha x) &= a (\alpha x) + b = \alpha (a x) + b \\[6pt]
            \alpha f(x) &= \alpha (a x + b) = \alpha (a x) + \alpha b
        \end{align}
        Z výpočtu vidíme, že ani jedna z vlastností neplatí. Ak vám niekto niekedy tvrdil, že funkcie typu (\ref{Linearity.Equation:Example3}) sú lineárne tak vám klamal. Napriek tomu je častokrát podstatné len správanie sa funkcií vzhľadom na deriváciu a funkcie (\ref{Linearity.Equation:Example1}) a (\ref{Linearity.Equation:Example3}) sa vzhľadom k derivácii správajú rovnako. Navyše voľbou vhodnej sústavy je možné funkciu (\ref{Linearity.Equation:Example3}) pretransformovať na lineárnu. Táto skupina funkcií sa nazýva afínne funkcie. Tento pojem je zovšeobecnením linearity. 
    \end{subequations}
\end{example}



\section{Afínne zobrazenie}
\label{Affinity}

\begin{definition}(Afínne zobrazenie)
    Nech $U$ a $V$ sú vektorové priestory nad poľom $K$. Hovoríme, že $f: V \rightarrow U$ je afínne zobrazenie ak pre ľubovoľné vektory $\bm{x}, \bm{y}, \bm{z} \in V$ a~ľubovoľný skalár $\alpha \in K$ platí:
    \begin{subequations}
        \begin{align}
            \label{Affinity.Equation:Aditivity}
            f(\bm{x} - \bm{y} + \bm{z})   &= f(\bm{x}) - f(\bm{y}) + f(\bm{z}) \\[6pt]
            \label{Affinity.Equation:Homogeneity}
            f(\alpha \bm{x} + (1 - \alpha) \bm{y}) &= \alpha f(\bm{x}) + (1 - \alpha) f(\bm{y})
        \end{align}
    \end{subequations}
\end{definition}



\end{document}
