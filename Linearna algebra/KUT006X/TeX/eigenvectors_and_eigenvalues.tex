\documentclass[a4paper, 10pt, ]{article}

\usepackage[slovak]{babel}

% ------------------------------

\usepackage[utf8]{inputenc}
\usepackage[T1]{fontenc}

\usepackage[left=4cm,
            right=4cm,
            top=2.1cm,
            bottom=2.6cm,
            footskip=7.5mm,
            twoside,
            marginparwidth=3.0cm,
            %showframe,
            ]{geometry}

\usepackage{graphicx}
\usepackage[dvipsnames]{xcolor}
% https://en.wikibooks.org/wiki/LaTeX/Colors

% ------------------------------

\usepackage{lmodern}

\usepackage[tt={oldstyle=false,proportional=true,monowidth}]{cfr-lm}
% https://mirror.szerverem.hu/ctan/fonts/cfr-lm/doc/cfr-lm.pdf

% ------------------------------

\usepackage{amsmath}
\usepackage{amssymb}
\usepackage{amsthm}

\usepackage{booktabs}
\usepackage{multirow}
\usepackage{array}
\usepackage{dcolumn}

\usepackage{natbib}

% ------------------------------

\hyphenpenalty=6000
\tolerance=1000

\def\naT{\mathsf{T}}

% ------------------------------

\makeatletter

    \def\@seccntformat#1{\protect\makebox[0pt][r]{\csname the#1\endcsname\hspace{4mm}}}

    \def\cleardoublepage{\clearpage\if@twoside \ifodd\c@page\else
    \hbox{}
    \vspace*{\fill}
    \begin{center}
    \phantom{}
    \end{center}
    \vspace{\fill}
    \thispagestyle{empty}
    \newpage
    \if@twocolumn\hbox{}\newpage\fi\fi\fi}

    \newcommand\figcaption{\def\@captype{figure}\caption}
    \newcommand\tabcaption{\def\@captype{table}\caption}

\makeatother

% ------------------------------

\usepackage{fancyhdr}
\fancypagestyle{plain}{%
\fancyhf{} % clear all header and footer fields
% \fancyfoot[C]{\sffamily {\bfseries \thepage}\ | {\scriptsize\oznacenieCasti}}
\fancyfoot[C]{\sffamily {\bfseries \thepage}{\color{Gray}\scriptsize$\,$z$\,$\pageref{LastPage}}\ | \includegraphics[height=5pt]{./COMMONFILES/KUT_logo_v0.1.pdf}{\scriptsize\KUTporadoveCislo}}
\renewcommand{\headrulewidth}{0pt}
\renewcommand{\footrulewidth}{0pt}}
\pagestyle{plain}

% ------------------------------

\usepackage{titlesec}
\titleformat{\paragraph}[hang]{\sffamily  \bfseries}{}{0pt}{}
\titlespacing*{\paragraph}{0mm}{3mm}{1mm}
\titlespacing*{\subparagraph}{0mm}{3mm}{1mm}

\titleformat*{\section}{\sffamily\Large\bfseries}
\titleformat*{\subsection}{\sffamily\large\bfseries}
\titleformat*{\subsubsection}{\sffamily\normalsize\bfseries}


% ------------------------------

\PassOptionsToPackage{hyphens}{url}
\usepackage[pdfauthor={},
            pdftitle={},
            pdfsubject={},
            pdfkeywords={},
            % hidelinks,
            colorlinks=false,
            breaklinks,
            ]{hyperref}


% ------------------------------

\graphicspath{%
{../fig_standalone/}%
{../../PY/fig/}%
{../../ML/fig/}%
{./fig/}%
}

% ------------------------------

\usepackage{enumitem}

\usepackage{lettrine}

% ------------------------------

\usepackage{lastpage}

\usepackage{microtype}

% ------------------------------

\usepackage{algorithm}
\usepackage[noend]{algpseudocode}
\makeatletter
\renewcommand{\ALG@name}{Algoritmus}
\makeatother
\usepackage{amsmath}
\usepackage{bbold}
\usepackage{calc}
\usepackage{dsfont}
\usepackage{mathtools}
\usepackage{tabto}


\newcommand{\mr}[1]{\mathrm{#1}}
\newcommand{\bs}[1]{\boldsymbol{#1}}
\newcommand{\bm}[1]{\mathbf{#1}}

\newcommand{\diff}[2]{\frac{\Delta #1}{\Delta #2}}
\newcommand{\der}[2]{\frac{d #1}{d #2}}
\newcommand{\parder}[2]{\frac{\partial #1}{\partial #2}}

\newcommand{\argmax}[0]{\mr{argmax}}
\newcommand{\diag}[0]{\mr{diag}}
\newcommand{\rank}[0]{\mr{rank}}
\newcommand{\trace}[0]{\mr{tr}}

\renewcommand{\Re}{\mr{Re}}
\renewcommand{\Im}{\mr{Im}}


\theoremstyle{definition}
\newtheorem{definition}{Definícia}[section]
\newtheorem{theorem}{Veta}[section]
\newtheorem{lemma}[theorem]{Lemma}
\newtheorem{example}{Príklad}[section]
\renewcommand*{\proofname}{Dôkaz}

% ------------------------------


% -----------------------------------------------------------------------------

\def\oznacenieCelku{Kolekcia učebných textov}

% -----------------------------------------------------------------------------


\def\KUTporadoveCislo{00X}

\def\oznacenieVerzie{v0.9}
% \def\oznacenieVerzie{\phantom{v1.0}}

\def\mesiacRok{apríl 2024}

\def\authorslabel{RM}





% -----------------------------------------------------------------------------

\begin{document}

% -----------------------------------------------------------------------------
% Uvodny nadpis

\noindent
\parbox[t][18mm][c]{0.3\textwidth}{%
\raisebox{-0.9\height}{%
\phantom{.}\includegraphics[height=18mm]{./COMMONFILES/URKFEIlogo.pdf}%
}%
}%
\parbox[t][18mm][c]{0.7\textwidth}{%
\raggedleft

\sffamily
\fontsize{16pt}{18pt}
\fontseries{sbc}
\selectfont

\noindent
\textcolor[rgb]{0.75, 0.75, 0.75}{\textls[25]{\oznacenieCelku}}
}%

\noindent
\parbox[t][16mm][b]{0.5\textwidth}{%
\raggedright

\color{Gray}
\sffamily

\fontsize{12pt}{12pt}
\selectfont
\mesiacRok

\fontsize{6pt}{10pt}
\selectfont
github.com/PracovnyBod/KUT

\fontsize{8pt}{10pt}
\selectfont
\authorslabel




}%
\parbox[t][16mm][b]{0.5\textwidth}{%
\raggedleft

\sffamily

\fontsize{6pt}{6pt}
\selectfont

\textcolor[rgb]{0.68, 0.68, 0.68}{\oznacenieVerzie}


\fontsize{14pt}{14pt}
\selectfont

\bfseries

\includegraphics[height=12pt]{./COMMONFILES/KUT_logo_v0.1.pdf}%
{%
\textls[-50]{\KUTporadoveCislo}
}%
}%

% -----------------------------------------------------------------------------




\vspace{6mm}

% ---------------------------------------------
\sffamily
\bfseries
\fontsize{18pt}{21pt}
\selectfont

\begin{flushleft}
    O vlastných vektoroch a vlastných číslach
\end{flushleft}

\bigskip

% -----------------------------------------------------------------------------
\normalsize
\normalfont
% -----------------------------------------------------------------------------

\section{Vlastné vektory a vlastné hodnoty}
\label{EigenvectorsEigenvalues}

\begin{definition}[Vlastné čísla matice]
    \label{EigenvectorsEigenvalues.Definition:Eigenvalues}
    Hovoríme, že nenulový vektor $\bm{v}$ je vlastný vektor štvorcovej matice $\bm{T}$, pokiaľ existuje skalár $\lambda$ taký, že:
    \begin{equation}
        \bm{T} \bm{v} = \lambda \bm{v}
    \end{equation}
    kde $\lambda$ nazývame vlastné číslo matice $\bm{T}$.
\end{definition}

Naskytuje sa teraz otázka ako také vlastné čísla a vlastné vektory nájsť. Čiže musíme vyriešiť rovnicu:
\begin{equation}
    \label{EigenvectorsEigenvalues.Equation:EigenvaluesEquation}
    \left( \bm{T} - \lambda \bm{I} \right) \bm{v} = \bm{0}
\end{equation}
kde $\bm{I}$ je jednotková matica.

Predpokladajme, že existuje inverzia matice $\left( \bm{T} - \lambda \bm{I} \right)$. Potom by malo platiť:
\begin{equation}
    \label{EigenvectorsEigenvalues.Equation:Contradiction}
    \bm{v} = \bm{I} \bm{v} = 
    \left( \bm{T} - \lambda \bm{I} \right)^{-1} \underbrace{\left( \bm{T} - \lambda \bm{I} \right) \bm{v}}_{\bm{0}} = \bm{0}
\end{equation}
Vidíme ale, že v (\ref{EigenvectorsEigenvalues.Equation:Contradiction}) sme prišli do sporu, lebo podľa definície \ref{EigenvectorsEigenvalues.Definition:Eigenvalues} je $\bm{v}$ nenulový vektor. Čiže matica $\left( \bm{T} - \lambda \bm{I} \right)$ nemôže byť invertibilná, to znamená, že je singulárna a pre jej determinant platí:
\begin{equation}
    \label{EigenvectorsEigenvalues.Equation:Determinant}
    \det \left\{ \bm{T} - \lambda \bm{I} \right\} = 0
\end{equation}
Vyriešením (\ref{EigenvectorsEigenvalues.Equation:Determinant}) teda môžeme získať vlastné čísla $\lambda$ matice $\bm{T}$.

\begin{example}
    \begin{subequations}
        Nájdite vlastné čísla a vlastné vektory matice:
        \begin{equation*}
            \bm{M} = 
            \begin{bmatrix}
                \phantom{-}3 & 7 \\
                          -4 & 1
            \end{bmatrix}
        \end{equation*}

        \noindent \textit{Vlastné čísla:}
        \begin{align*}
            &\det \left\{ \bm{M} - \lambda \bm{I} \right\} = 
            \det \left\{ 
                \begin{bmatrix}
                    4 & \phantom{-}7 \\
                    1 & -2
                \end{bmatrix} - 
                \begin{bmatrix}
                    \lambda & 0      \\
                    0       & \lambda
                \end{bmatrix}
            \right\} = 
            \det \left\{
                \begin{bmatrix}
                    4 - \lambda & \phantom{-}7 \\
                    1           & -2 - \lambda
                \end{bmatrix}
            \right\} = 0 \\
            &\det \left\{
                \begin{bmatrix}
                    4 - \lambda & \phantom{-}7 \\
                    1           & -2 - \lambda
                \end{bmatrix}
            \right\} = 
            (4 - \lambda) (-2 - \lambda) - (1 \cdot 7) = \lambda^2 - 2\lambda - 15 \\
            &\lambda^2 - 2\lambda - 15 = 0 \ \Longrightarrow \ \lambda_1 = 5, \ \lambda_2 = -3
        \end{align*}

        \noindent \textit{Vlastné vektory:}

        $\lambda = 5$:
        \begin{align*}
            &\left( \bm{M} - \lambda \bm{I} \right) \bm{v} = 0 \ \thicksim \ 
            \begin{bmatrix}
                4 - 5 & 7 \\
                1     & -2 - 5
            \end{bmatrix}
            \begin{bmatrix}
                v_1 \\ 
                v_2
            \end{bmatrix} = 
            \begin{bmatrix}
                0 \\ 
                0
            \end{bmatrix} \\
            &\begin{bmatrix}
                -1           & \phantom{-}7 \\
                \phantom{-}1 & -7
            \end{bmatrix}
            \begin{bmatrix}
                v_1 \\ 
                v_2
            \end{bmatrix} = 
            \begin{bmatrix}
                0 \\ 
                0
            \end{bmatrix} \thicksim 
            \begin{bmatrix}
                -1           & 7 \\
                \phantom{-}0 & 0
            \end{bmatrix} 
            \begin{bmatrix}
                v_1 \\ 
                v_2
            \end{bmatrix} = 
            \begin{bmatrix}
                0 \\ 
                0
            \end{bmatrix}
        \end{align*}
        
        \noindent Sústava rovníc má nekonečne veľa riešení, riešenie je parametrizované parametrom $t$, ktorý môže byť ľubovoľné reálne číslo okrem $0$.
        \begin{align*}
            &v_2 = t, \quad t \in \mathbb{R} \setminus \{0\} \\
            &v_1 = 7 v_2 = 7 t \\
        \end{align*}
        Vlastný vektor pre $\lambda = 5$ je napríklad:
        \begin{equation*}    
            \bm{v} = 
            \begin{bmatrix}
                7t \\
                t
            \end{bmatrix} = 
            \begin{bmatrix}
                7 \\
                1
            \end{bmatrix}, \quad t = 1
        \end{equation*}
        

        $\lambda = 5$:
        \begin{align*}
            &\left( \bm{M} - \lambda \bm{I} \right) \bm{v} = 0 \ \thicksim \ 
            \begin{bmatrix}
                4 + 3 & 7 \\
                1     & -2 + 3
            \end{bmatrix}
            \begin{bmatrix}
                v_1 \\ 
                v_2
            \end{bmatrix} = 
            \begin{bmatrix}
                0 \\ 
                0
            \end{bmatrix} \\
            &\begin{bmatrix}
                7 & 7 \\
                1 & 1
            \end{bmatrix}
            \begin{bmatrix}
                v_1 \\ 
                v_2
            \end{bmatrix} = 
            \begin{bmatrix}
                0 \\ 
                0
            \end{bmatrix} \thicksim 
            \begin{bmatrix}
                7 & 7 \\
                0 & 0
            \end{bmatrix} 
            \begin{bmatrix}
                v_1 \\ 
                v_2
            \end{bmatrix} = 
            \begin{bmatrix}
                0 \\ 
                0
            \end{bmatrix}
        \end{align*}
    
        \noindent Sústava rovníc má nekonečne veľa riešení, riešenie je parametrizované parametrom $t$, ktorý môže byť ľubovoľné reálne číslo okrem $0$.
        \begin{align*}
            &v_2 = t, \quad t \in \mathbb{R} \setminus \{0\} \\
            &v_1 = v_2 = 7 \\
        \end{align*}
         Vlastný vektor pre $\lambda = -3$ je napríklad:
        \begin{equation*}    
            \bm{v} = 
            \begin{bmatrix}
                t \\
                t
            \end{bmatrix} = 
            \begin{bmatrix}
                1 \\
                1
            \end{bmatrix}, \quad t = 1
        \end{equation*}
    \end{subequations}
\end{example}

\end{document}
