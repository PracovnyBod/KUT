\documentclass[a4paper, 10pt, ]{article}

\usepackage[slovak]{babel}

% ------------------------------

\usepackage[utf8]{inputenc}
\usepackage[T1]{fontenc}

\usepackage[left=4cm,
            right=4cm,
            top=2.1cm,
            bottom=2.6cm,
            footskip=7.5mm,
            twoside,
            marginparwidth=3.0cm,
            %showframe,
            ]{geometry}

\usepackage{graphicx}
\usepackage[dvipsnames]{xcolor}
% https://en.wikibooks.org/wiki/LaTeX/Colors

% ------------------------------

\usepackage{lmodern}

\usepackage[tt={oldstyle=false,proportional=true,monowidth}]{cfr-lm}
% https://mirror.szerverem.hu/ctan/fonts/cfr-lm/doc/cfr-lm.pdf

% ------------------------------

\usepackage{amsmath}
\usepackage{amssymb}
\usepackage{amsthm}

\usepackage{booktabs}
\usepackage{multirow}
\usepackage{array}
\usepackage{dcolumn}

\usepackage{natbib}

% ------------------------------

\hyphenpenalty=6000
\tolerance=1000

\def\naT{\mathsf{T}}

% ------------------------------

\makeatletter

    \def\@seccntformat#1{\protect\makebox[0pt][r]{\csname the#1\endcsname\hspace{4mm}}}

    \def\cleardoublepage{\clearpage\if@twoside \ifodd\c@page\else
    \hbox{}
    \vspace*{\fill}
    \begin{center}
    \phantom{}
    \end{center}
    \vspace{\fill}
    \thispagestyle{empty}
    \newpage
    \if@twocolumn\hbox{}\newpage\fi\fi\fi}

    \newcommand\figcaption{\def\@captype{figure}\caption}
    \newcommand\tabcaption{\def\@captype{table}\caption}

\makeatother

% ------------------------------

\usepackage{fancyhdr}
\fancypagestyle{plain}{%
\fancyhf{} % clear all header and footer fields
% \fancyfoot[C]{\sffamily {\bfseries \thepage}\ | {\scriptsize\oznacenieCasti}}
\fancyfoot[C]{\sffamily {\bfseries \thepage}{\color{Gray}\scriptsize$\,$z$\,$\pageref{LastPage}}\ | \includegraphics[height=5pt]{./COMMONFILES/KUT_logo_v0.1.pdf}{\scriptsize\KUTporadoveCislo}}
\renewcommand{\headrulewidth}{0pt}
\renewcommand{\footrulewidth}{0pt}}
\pagestyle{plain}

% ------------------------------

\usepackage{titlesec}
\titleformat{\paragraph}[hang]{\sffamily  \bfseries}{}{0pt}{}
\titlespacing*{\paragraph}{0mm}{3mm}{1mm}
\titlespacing*{\subparagraph}{0mm}{3mm}{1mm}

\titleformat*{\section}{\sffamily\Large\bfseries}
\titleformat*{\subsection}{\sffamily\large\bfseries}
\titleformat*{\subsubsection}{\sffamily\normalsize\bfseries}


% ------------------------------

\PassOptionsToPackage{hyphens}{url}
\usepackage[pdfauthor={},
            pdftitle={},
            pdfsubject={},
            pdfkeywords={},
            % hidelinks,
            colorlinks=false,
            breaklinks,
            ]{hyperref}


% ------------------------------

\graphicspath{%
{../fig_standalone/}%
{../../PY/fig/}%
{../../ML/fig/}%
{./fig/}%
}

% ------------------------------

\usepackage{enumitem}

\usepackage{lettrine}

% ------------------------------

\usepackage{lastpage}

\usepackage{microtype}

% ------------------------------

\usepackage{algorithm}
\usepackage[noend]{algpseudocode}
\makeatletter
\renewcommand{\ALG@name}{Algoritmus}
\makeatother
\usepackage{amsmath}
\usepackage{bbold}
\usepackage{calc}
\usepackage{dsfont}
\usepackage{mathtools}
\usepackage{tabto}


\newcommand{\mr}[1]{\mathrm{#1}}
\newcommand{\bs}[1]{\boldsymbol{#1}}
\newcommand{\bm}[1]{\mathbf{#1}}

\newcommand{\diff}[2]{\frac{\Delta #1}{\Delta #2}}
\newcommand{\der}[2]{\frac{d #1}{d #2}}
\newcommand{\parder}[2]{\frac{\partial #1}{\partial #2}}

\newcommand{\argmax}[0]{\mr{argmax}}
\newcommand{\diag}[0]{\mr{diag}}
\newcommand{\rank}[0]{\mr{rank}}
\newcommand{\trace}[0]{\mr{tr}}

\renewcommand{\Re}{\mr{Re}}
\renewcommand{\Im}{\mr{Im}}


\theoremstyle{definition}
\newtheorem{definition}{Definícia}[section]
\newtheorem{theorem}{Veta}[section]
\newtheorem{lemma}[theorem]{Lemma}
\newtheorem{example}{Príklad}[section]
\renewcommand*{\proofname}{Dôkaz}

% ------------------------------


% -----------------------------------------------------------------------------

\def\oznacenieCelku{Kolekcia učebných textov}

% -----------------------------------------------------------------------------


\def\KUTporadoveCislo{002}

% \def\oznacenieVerzie{v1.0}
\def\oznacenieVerzie{\phantom{v1.0}}

\def\mesiacRok{marec 2024}

\def\authorslabel{MT}





% -----------------------------------------------------------------------------

\begin{document}

% -----------------------------------------------------------------------------
% Uvodny nadpis

\noindent
\parbox[t][18mm][c]{0.3\textwidth}{%
\raisebox{-0.9\height}{%
\phantom{.}\includegraphics[height=18mm]{./COMMONFILES/URKFEIlogo.pdf}%
}%
}%
\parbox[t][18mm][c]{0.7\textwidth}{%
\raggedleft

\sffamily
\fontsize{16pt}{18pt}
\fontseries{sbc}
\selectfont

\noindent
\textcolor[rgb]{0.75, 0.75, 0.75}{\textls[25]{\oznacenieCelku}}
}%

\noindent
\parbox[t][16mm][b]{0.5\textwidth}{%
\raggedright

\color{Gray}
\sffamily

\fontsize{12pt}{12pt}
\selectfont
\mesiacRok

\fontsize{6pt}{10pt}
\selectfont
github.com/PracovnyBod/KUT

\fontsize{8pt}{10pt}
\selectfont
\authorslabel




}%
\parbox[t][16mm][b]{0.5\textwidth}{%
\raggedleft

\sffamily

\fontsize{6pt}{6pt}
\selectfont

\textcolor[rgb]{0.68, 0.68, 0.68}{\oznacenieVerzie}


\fontsize{14pt}{14pt}
\selectfont

\bfseries

\includegraphics[height=12pt]{./COMMONFILES/KUT_logo_v0.1.pdf}%
{%
\textls[-50]{\KUTporadoveCislo}
}%
}%

% -----------------------------------------------------------------------------




\vspace{6mm}

% ---------------------------------------------
\sffamily
\bfseries
\fontsize{18pt}{21pt}
\selectfont

\begin{flushleft}
	O~stave a~začiatočných podmienkach dynamického systému
\end{flushleft}

\bigskip

% -----------------------------------------------------------------------------
\normalsize
\normalfont
% -----------------------------------------------------------------------------












\noindent
\lettrine[lines=1, nindent=1pt, loversize=0.0]{S}{tav} 
systému a~začiatočné podmienky systému sú relatívne široké pojmy. Cieľom je tu načrtnúť pohľad klasickej Teórie systémov.



\paragraph{Zotrvačnosť}

Ak hovoríme o~dynamickom systéme, obvykle máme na mysli \emph{zotrvačný} systém.

\emph{Bezzotrvačný} systém je taký, ktorého výstup závisí len od okamžitej hodnoty vstupu. Minulé hodnoty vstupu nemajú vplyv na aktuálnu hodnotu výstupu. Príkladom bezzotrvačného systému môže byť odporový delič napätia.

O~\emph{zotrvačnom} systéme možno premýšľať ako o~systéme „s pamäťou“. Jeho aktuálny výstup závisí od okamžitej hodnoty vstupu a~aj od hodnôt vstupu v~minulosti. Takýto systém je \emph{kauzálny}.


Mimochodom, teoreticky je možné hovoriť aj o~nekauzálnych systémoch. v~takom prípade výstup závisí aj od budúcich hodnôt vstupu. Slová \emph{zotrvačnosť} a~\emph{kauzalita} tu odporúčame vnímať najmä z fyzikálneho hľadiska.

Príkladom zotrvačného kauzálneho systému môže byť elektrický kondenzátor. Aktuálny elektrický náboj $Q(t)$ na kondenzátore je daný takpovediac celou „históriou“ elektrického prúdu $I(t)$ cez kondenzátor. Formálne, náboj v~čase $t_0$ bude
\begin{equation}
    Q(t_0) = \int_{-\infty}^{t_0}I(t) \text{d}t
\end{equation}


\paragraph{Stav systému}

Predošlý príklad navádza na otázky typu ako ďaleko do minulosti ešte ovplyvňujú hodnoty vstupu terajší výstup? Je azda potrebné poznať takpovediac úplnú minulosť systému? Ukazuje sa tu pojem \emph{stav systému}.

Spomínaný náboj $Q$ na kondenzátore má v~čase $t_0$ nejakú hodnotu. Označme ju~$Q_0$. Kondenzátor ako systém je teda v~takom stave, že jeho veličina má hodnotu~$Q_0$. Je to tak presne v~čase $t_0$. 

Ak by sme prišli k tomuto systému v~čase $t_0$, vedeli by sme zistiť, že je v~nejakom stave. v~tomto prípade vyjadriteľnom hodnotou~$Q_0$. Ak by sme chceli zmeniť hodnotu náboja na kondenzátore, teda to v~akom je stave, musíme nejaký čas pôsobiť na vstupe tohto systému (priviesť napätie na svorky kondenzátora).

Poznáme stav systému na začiatku nášho pôsobenia na vstupe a~následne na konci pôsobenia bude vo všeobecnosti systém v~inom stave ako na začiatku. Systém je v~každom čase v~nejakom stave.

Na rozdiel od značne intuitívneho stavu systému v~prípade kondenzátora, určiť ktoré veličiny udávajú stav systému a~či ich vieme merať je samostatná otázka.






\paragraph{Príklad s mechanickým systémom}

% Poznámka: na tento príklad nadväzujeme aj v~ďalšej časti textu. \medskip

Uvažujme posuvný pohyb telesa s hmotnosťou $m$, na ktoré pôsobí sila $u(t)$. Vo všeobecnosti časovo premenlivá sila je vstupom systému. Teleso sa pohybuje po priamke. Poloha tohto telesa, teda jeho vzdialenosť od stanoveného bodu nula nech je výstupnou veličinou systému. Označme $y(t)$.

\begin{center}

    \vbox{%
        \makebox[\textwidth][c]{%
        \input{../SVG/mechTeleso.pdf_tex}
        }
        \figcaption{Posuvný pohyb telesa s hmotnosťou $m$.}
    }
	\label{mechTeleso}

\end{center}

Pohybová rovnica v~tomto prípade je\footnote{Newtonove $F = ma$}
\begin{equation} \label{pohybr}
    m \ddot y(t) = u(t)
\end{equation}
Opisuje uvedenú situáciu ako dynamický systém. Dáva do vzťahu vstup a~výstup systému. Je to diferenciálna rovnica druhého rádu.

Ako však určiť v~akom stave je systém? 

Čo potrebujeme poznať, aby sme dokázali určiť výstup systému od nejakého začiatočného času $t_0$ ďalej, teda na intervale $\langle t_0, t \rangle$?

Poloha, výstup systému, je daná začiatočnou polohou v~čase $t_0$, označme $y_0$, a~ďalej je daná integrálom rýchlosti telesa $\dot y(t)$ na uvedenom intervale.
\begin{equation}
    y(t) = y_0 + \int_{t_0}^{t}\dot y(\tau) \text{d}\tau
\end{equation}
kde samozrejme $\tau \in \langle t_0, t \rangle$.

Čím je však potom daný priebeh rýchlosti $\dot y(t)$? Na intervale $\langle t_0, t \rangle$ zjavne musí platiť
\begin{equation}
    \dot y(t) = z_0 + \int_{t_0}^{t}\ddot y(\tau) \text{d}\tau
\end{equation}
kde $z_0$ je začiatočná rýchlosť telesa v~čase $t_0$ a~$\ddot y(t)$ je časový priebeh zrýchlenia telesa. 

Poznáme časový priebeh zrýchlenia na intervale $\langle t_0, t \rangle$? Áno. Priamo vďaka diferenciálnej rovnici opisujúcej samotný systém. Z rovnice \eqref{pohybr} priamo vyplýva
\begin{equation}
    \dot y(t) = \frac{1}{m} \int_{t_0}^{t}  u(\tau) \text{d}\tau
\end{equation}
Časový priebeh vstupného signálu $u(t)$ vo všeobecnosti poznáme. Vstupnú veličinu máme v~moci my.

Na určenie výstupu systému v~čase $t$ sme teda potrebovali poznať nasledovné: začiatočnú polohu, začiatočnú rýchlosť a~časový priebeh vstupnej veličiny od začiatku po čas $t$.

Stav systému na začiatku ale nie len na začiatku je daný dvomi veličinami. Polohou a~rýchlosťou. Ich hodnoty je možné považovať za stav systému. Tieto veličiny možno označiť ako stavové veličiny systému.

Stav systému sa zmení ak nejaký čas bude pôsobiť vstupný signál.


\paragraph{Stavový vektor}

Stavových veličín je vo všeobecnosti niekoľko. Minimálne toľko akého rádu je diferenciálna rovnica opisujúca dynamický systém. Hovoríme tak o~ráde systému. V~predchádzajúcom príklade je systém  druhého rádu, $n=2$.

Je praktické usporiadať stavové veličiny do stavového vektora. V~predchádzajúcom príklade sa stavový vektor skladá z dvoch stavových veličín. Stavový vektor označme $x(t)$, formálne sa jeho rozmer uvádza ako $x(t) \in \mathbb R^n$.
\begin{equation}
    x(t) = 
    \begin{bmatrix}
        y(t) \\ \dot y(t)
    \end{bmatrix}
\end{equation}
kde sme uviedli stavové veličiny z predchádzajúceho príkladu. Zvyčajne sa stavové veličiny označujú samostatnými symbolmi, typicky
\begin{equation}
    x(t) = 
    \begin{bmatrix}
        x_1(t) \\ x_2(t)
    \end{bmatrix}
\end{equation}

Prvky stavového vektora $x(t)$ môžu nadobúdať rôzne hodnoty, inými slovami vektor $x(t)$ je v~nejakom priestore, hovoríme o~\emph{stavovom priestore}.


\paragraph{Opis systému v~stavovom priestore}

Okrem diferenciálnej rovnice, v~ktorej figurujú vstup a~výstup systému, je možné systém opísať aj sústavou diferenciálnych rovníc, v~ktorých  figurujú stavové veličiny. Ide o~sústavu sústavu diferenciálnych rovníc prvého rádu, v~ktorých neznámou premenou je stavová veličina. Hovoríme o~opise systému v~stavovom priestore.

V~nadväznosti na predchádzajúci príklad uvážme opis mechanického systému v~stavovom priestore. Stavové veličiny máme označené $x_1$ a~$x_2$. Hľadáme sústavu diferenciálnych rovníc
\begin{align*}
    \dot x_1(t) &= \quad ? \\
    \dot x_2(t) &= \quad ? 
\end{align*}
pričom na pravej strane rovníc môžu vystupovať len stavové veličiny a~vstupný signál.

Veličina $x_1(t)$ je poloha. Veličina $x_2(t)$ je jej časová derivácia, teda rýchlosť. Časová derivácia rýchlosti je zrýchlenie a~zo samotnej pohybovej rovnice plynie, že zrýchlenie sa rovná výrazu $\frac{1}{m} u(t)$, teda je dané vstupom systému. Takže
\begin{subequations}
    \begin{align}
        \dot x_1(t) &= x_2(t) \\
        \dot x_2(t) &= \frac{1}{m} u(t)
    \end{align}    
\end{subequations}
je sústava diferenciálnych rovníc prvého rádu opisujúca predmetný dynamický systém. Vzťah výstupu systému a~stavových veličín je pri tom tiež jasný
\begin{equation}
    y(t) = x_1(t)
\end{equation}

Ak uvážime stavový vektor $x(t)$, uvedené je možné zapísať vo vektorovo-maticovom tvare
\begin{subequations}
    \begin{align}
        \begin{bmatrix}
            \dot x_1(t) \\ \dot x_2(t)
        \end{bmatrix}
        & =
        \begin{bmatrix}
            0 & 1 \\ 0 & 0
        \end{bmatrix}
        \begin{bmatrix}
            x_1(t) \\ x_2(t)
        \end{bmatrix}
        +
        \begin{bmatrix}
            0 \\ \frac{1}{m}
        \end{bmatrix}
        u(t)
        \\
        y(t)
        &=
        \begin{bmatrix}
            1 & 0
        \end{bmatrix}
        \begin{bmatrix}
            x_1(t) \\ x_2(t)
        \end{bmatrix}
    \end{align}
\end{subequations}

V~takýchto prípadoch sa opis systému v~stavovom priestore zapisuje v~tvare
\begin{subequations}
    \begin{align}
        \dot x(t) &= A x(t) + b u(t) \\
        y(t) &= c^\naT x(t)
    \end{align}
\end{subequations}
kde $A$ je matica, $b$ a~$c$ sú vektory. V~predchádzajúcom príklade $A \in \mathbb R^{2\times 2}$, $b \in \mathbb R^{2}$ a~$c \in \mathbb R^{2}$. 








\nocite{*}

\printbibliography[title={Odporúčaná literatúra}]



% -----------------------------------------------------------------------------

\end{document}

% -----------------------------------------------------------------------------