\documentclass[a4paper, 10pt, ]{article}

\usepackage[slovak]{babel}

% ------------------------------

\usepackage[utf8]{inputenc}
\usepackage[T1]{fontenc}

\usepackage[left=4cm,
            right=4cm,
            top=2.1cm,
            bottom=2.6cm,
            footskip=7.5mm,
            twoside,
            marginparwidth=3.0cm,
            %showframe,
            ]{geometry}

\usepackage{graphicx}
\usepackage[dvipsnames]{xcolor}
% https://en.wikibooks.org/wiki/LaTeX/Colors

% ------------------------------

\usepackage{lmodern}

\usepackage[tt={oldstyle=false,proportional=true,monowidth}]{cfr-lm}
% https://mirror.szerverem.hu/ctan/fonts/cfr-lm/doc/cfr-lm.pdf

% ------------------------------

\usepackage{amsmath}
\usepackage{amssymb}
\usepackage{amsthm}

\usepackage{booktabs}
\usepackage{multirow}
\usepackage{array}
\usepackage{dcolumn}

\usepackage{natbib}

% ------------------------------

\hyphenpenalty=6000
\tolerance=1000

\def\naT{\mathsf{T}}

% ------------------------------

\makeatletter

    \def\@seccntformat#1{\protect\makebox[0pt][r]{\csname the#1\endcsname\hspace{4mm}}}

    \def\cleardoublepage{\clearpage\if@twoside \ifodd\c@page\else
    \hbox{}
    \vspace*{\fill}
    \begin{center}
    \phantom{}
    \end{center}
    \vspace{\fill}
    \thispagestyle{empty}
    \newpage
    \if@twocolumn\hbox{}\newpage\fi\fi\fi}

    \newcommand\figcaption{\def\@captype{figure}\caption}
    \newcommand\tabcaption{\def\@captype{table}\caption}

\makeatother

% ------------------------------

\usepackage{fancyhdr}
\fancypagestyle{plain}{%
\fancyhf{} % clear all header and footer fields
% \fancyfoot[C]{\sffamily {\bfseries \thepage}\ | {\scriptsize\oznacenieCasti}}
\fancyfoot[C]{\sffamily {\bfseries \thepage}{\color{Gray}\scriptsize$\,$z$\,$\pageref{LastPage}}\ | \includegraphics[height=5pt]{./COMMONFILES/KUT_logo_v0.1.pdf}{\scriptsize\KUTporadoveCislo}}
\renewcommand{\headrulewidth}{0pt}
\renewcommand{\footrulewidth}{0pt}}
\pagestyle{plain}

% ------------------------------

\usepackage{titlesec}
\titleformat{\paragraph}[hang]{\sffamily  \bfseries}{}{0pt}{}
\titlespacing*{\paragraph}{0mm}{3mm}{1mm}
\titlespacing*{\subparagraph}{0mm}{3mm}{1mm}

\titleformat*{\section}{\sffamily\Large\bfseries}
\titleformat*{\subsection}{\sffamily\large\bfseries}
\titleformat*{\subsubsection}{\sffamily\normalsize\bfseries}


% ------------------------------

\PassOptionsToPackage{hyphens}{url}
\usepackage[pdfauthor={},
            pdftitle={},
            pdfsubject={},
            pdfkeywords={},
            % hidelinks,
            colorlinks=false,
            breaklinks,
            ]{hyperref}


% ------------------------------

\graphicspath{%
{../fig_standalone/}%
{../../PY/fig/}%
{../../ML/fig/}%
{./fig/}%
}

% ------------------------------

\usepackage{enumitem}

\usepackage{lettrine}

% ------------------------------

\usepackage{lastpage}

\usepackage{microtype}

% ------------------------------

\usepackage{algorithm}
\usepackage[noend]{algpseudocode}
\makeatletter
\renewcommand{\ALG@name}{Algoritmus}
\makeatother
\usepackage{amsmath}
\usepackage{bbold}
\usepackage{calc}
\usepackage{dsfont}
\usepackage{mathtools}
\usepackage{tabto}


\newcommand{\mr}[1]{\mathrm{#1}}
\newcommand{\bs}[1]{\boldsymbol{#1}}
\newcommand{\bm}[1]{\mathbf{#1}}

\newcommand{\diff}[2]{\frac{\Delta #1}{\Delta #2}}
\newcommand{\der}[2]{\frac{d #1}{d #2}}
\newcommand{\parder}[2]{\frac{\partial #1}{\partial #2}}

\newcommand{\argmax}[0]{\mr{argmax}}
\newcommand{\diag}[0]{\mr{diag}}
\newcommand{\rank}[0]{\mr{rank}}
\newcommand{\trace}[0]{\mr{tr}}

\renewcommand{\Re}{\mr{Re}}
\renewcommand{\Im}{\mr{Im}}


\theoremstyle{definition}
\newtheorem{definition}{Definícia}[section]
\newtheorem{theorem}{Veta}[section]
\newtheorem{lemma}[theorem]{Lemma}
\newtheorem{example}{Príklad}[section]
\renewcommand*{\proofname}{Dôkaz}

% ------------------------------


% -----------------------------------------------------------------------------

\def\oznacenieCelku{Kolekcia učebných textov}

% -----------------------------------------------------------------------------


\def\KUTporadoveCislo{003}

\def\oznacenieVerzie{v0.9}
% \def\oznacenieVerzie{\phantom{v1.0}}

\def\mesiacRok{apríl 2024}

\def\authorslabel{MT}





% -----------------------------------------------------------------------------

\begin{document}

% -----------------------------------------------------------------------------
% Uvodny nadpis

\noindent
\parbox[t][18mm][c]{0.3\textwidth}{%
\raisebox{-0.9\height}{%
\phantom{.}\includegraphics[height=18mm]{./COMMONFILES/URKFEIlogo.pdf}%
}%
}%
\parbox[t][18mm][c]{0.7\textwidth}{%
\raggedleft

\sffamily
\fontsize{16pt}{18pt}
\fontseries{sbc}
\selectfont

\noindent
\textcolor[rgb]{0.75, 0.75, 0.75}{\textls[25]{\oznacenieCelku}}
}%

\noindent
\parbox[t][16mm][b]{0.5\textwidth}{%
\raggedright

\color{Gray}
\sffamily

\fontsize{12pt}{12pt}
\selectfont
\mesiacRok

\fontsize{6pt}{10pt}
\selectfont
github.com/PracovnyBod/KUT

\fontsize{8pt}{10pt}
\selectfont
\authorslabel




}%
\parbox[t][16mm][b]{0.5\textwidth}{%
\raggedleft

\sffamily

\fontsize{6pt}{6pt}
\selectfont

\textcolor[rgb]{0.68, 0.68, 0.68}{\oznacenieVerzie}


\fontsize{14pt}{14pt}
\selectfont

\bfseries

\includegraphics[height=12pt]{./COMMONFILES/KUT_logo_v0.1.pdf}%
{%
\textls[-50]{\KUTporadoveCislo}
}%
}%

% -----------------------------------------------------------------------------




\vspace{6mm}

% ---------------------------------------------
\sffamily
\bfseries
\fontsize{18pt}{21pt}
\selectfont

\begin{flushleft}
	O regulačnom obvode
\end{flushleft}

\bigskip

% -----------------------------------------------------------------------------
\normalsize
\normalfont
% -----------------------------------------------------------------------------












\noindent
\lettrine[lines=1, nindent=1pt, loversize=0.0]{R}{egulačný} 
obvod sa vo všeobecnosti skladá z riadiaceho systému a~z riadeného systému. Zahŕňa tri základné signály. Výstupnú veličinu $y(t)$, akčný zásah $u(t)$ a~referenčný signál $r(t)$. 



\section{Všeobecný uzavretý regulačný obvod}

Schematicky sa všeobecný uzavretý regulačný obvod znázorňuje nasledovne:

\begin{center}

    \vbox{%
    \makebox[\textwidth][c]{%
	\input{../SVG/schURO_vseob.pdf_tex}
	}

    % \vspace{-2mm}

	\figcaption{Všeobecný uzavretý regulačný obvod.}
    \label{schURO_vseob.pdf}
    }

\end{center}

Výstupom riadeného systému je veličina, ktorá, okrem iného, hovorí o~splnení cieľa riadenia. Cieľom riadenia napríklad je, aby táto veličina dosiahla istú hodnotu, prípadne aby priebeh tejto veličiny v~čase vykazoval isté dynamické vlastnosti, a~podobne. Pre skrátenie sa práve táto veličina nazýva ako výstupná veličina (celého obvodu). Označuje sa $y(t)$.

Úlohou riadiaceho systému je splniť cieľ riadenia. Výstupom riadiaceho systému je tzv. akčný zásah (označuje sa $u(t)$). Je to signál (veličina), pomocou ktorého riadiaci systém ovplyvňuje riadený systém. Akčný zásah je teda na vstupe riadeného systému.

Pre splnenie cieľa riadenia potrebuje riadiaci systém dostať príkaz typicky vo forme signálu, ktorý je referenčným signálom (označuje sa $r(t)$) alebo žiadanou hodnotou (označuje sa $w(t)$, v angličtine \emph{setpoint}). Druhou informáciou, ktorú riadiaci systém potrebuje pre splnenie cieľa, je spätná väzba z výstupu riadeného systému.

S využitím uvedeného, teda spätnej väzby a~referenčného signálu (alebo žiadanej hodnoty), riadiaci systém akčným zásahom ovplyvňuje riadiaci systém tak, aby bol splnený cieľ riadenia. Pre zvýraznenie princípu spätnej väzby sa výsledný principiálny regulačný obvod nazýva \emph{Uzavretý regulačný obvod} (URO).




\section{Regulačná odchýlka}

Značne typickým uzavretým regulačným obvodom je taký, v ktorom sa využíva regulačná odchýlka.

Regulačná odchýlka $e(t)$ je rozdiel žiadanej hodnoty $w(t)$ (setpoint) a~výstupnej veličiny riadeného systému $y(t)$, teda
\begin{align} \label{regodch}
    e(t) = w(t) - y(t)
\end{align}
Je zrejmé, že ak je regulačná odchýlka nulová, tak cieľ riadenia je splnený.

V tomto prípade je URO možné schematicky znázorniť nasledovne:

\begin{center}

    \vbox{%
    \makebox[\textwidth][c]{%
	\input{../SVG/schURO_regodch.pdf_tex}
	}

	\figcaption{Uzavretý regulačný obvod s regulačnou odchýlkou a~regulátorom.}
    \label{schURO_regodch.pdf}
    }
	

\end{center}

Je možné konštatovať, že na obr.~\ref{schURO_regodch.pdf} je celkový riadiaci systém tvorený dvomi prvkami: výpočtom regulačnej odchýlky a~regulátorom. Typicky, vstupom regulátora je regulačná odchýlka.

Pre zdôraznenie faktu, že v uvedenom prípade ide jednoznačne (už z princípu informácie o~odchýlke \eqref{regodch}) o~zápornú spätnú väzbu môžeme kresliť schému nasledovne:



\begin{center}

    \vbox{%
    \makebox[\textwidth][c]{%
	\input{../SVG/schURO_regodchminus.pdf_tex}
	}

    \vspace{-2mm}

	\figcaption{Uzavretý regulačný obvod s blokom vyjadrujúcim zápornú spätnú väzbu.}
	\label{schURO_regodchminus.pdf}
    }

\end{center}




\section{Lineárny uzavretý regulačný obvod}

V prípade, že riadiaci a~riadený systém je možné opísať ako lineárne dynamické systémy, hovoríme o~lineárnom uzavretom regulačnom obvode.

Typicky hovoríme, že regulátor, ktorého vstupom je regulačná odchýlka, a~riadený systém je vtedy možné reprezentovať prenosovými funkciami. Klasický lineárny uzavretý regulačný obvod je potom možné schematicky znázorniť nasledovne:

\begin{center}

    \vbox{%
    \makebox[\textwidth][c]{%
	\input{../SVG/schLinURO_basic.pdf_tex}
	}

    \vspace{-3mm}

	\figcaption{Lineárny uzavretý regulačný obvod.}
    \label{schLinURO_basic.pdf}
    }
	

\end{center}


V tomto prípade všetky bloky v schéme sú tvorené prenosovými funkciami (aj $-1$ je v princípe prenosová funkcia) pričom $G_R(s)$ je prenosová funkcia regulátora a~$G_S(s)$ je prenosová funkcia riadeného systému (hovorí sa tiež prenosová funkcia riadenej sústavy).

Avšak, ak sú blokmi URO prenosové funkcie, potom namiesto časových signálov je možné uvažovať ich Laplaceove obrazy (L-obrazy), teda $W(s)$, $E(s)$, $U(s)$ a~$Y(s)$.


\subsection{Otvorený regulačný obvod}

S využitím algebry prenosových funkcií vidíme, že $G_R(s)$ a~$G_S(s)$ sú v sérii a~teda máme
\begin{align}
    G_{ORO}(s) = G_R(s) G_S(s)
\end{align}
pričom $G_{ORO}(s)$ je prenosová funkcia súvisiaca s pojmom \emph{otvorený regulačný obvod} -- je to situácia keď sa neuvažuje spätná väzba (obvod nie je uzavretý).

\begin{center}

    \vspace{-2mm}

    \vbox{%
    \makebox[\textwidth][c]{%
	\input{../SVG/schLinORO_basic.pdf_tex}
	}

    \vspace{-3mm}

	\figcaption{Otvorený regulačný obvod.}
    \label{schLinORO_basic.pdf}
    }
	

\end{center}




\subsection{Prenosová funkcia URO}

Zároveň, na URO je potom jednoduché pozrieť sa ako na jeden celok. Celkovým výstupom URO je výstupná veličina $y(t)$, ktorej L-obraz je $Y(s)$, a~celkovým vstupom URO je žiadaná hodnota $w(t)$ s obrazom $W(s)$. Pomer obrazov $W(s)$ a~$Y(s)$ je prenosovou funkciou URO.
\begin{align}
    \frac{Y(s)}{W(s)} = G_{URO}(s)
\end{align}

Prenosovú funkciu URO je ďalej možné konkretizovať s využitím algebry prenosových funkcií. S využitím $G_{ORO}(s)$ máme situáciu:

\begin{center}

    \vspace{-1mm}

    \vbox{%
    \makebox[\textwidth][c]{%
	\input{../SVG/schLinURO_basic_soro.pdf_tex}
	}

    \vspace{-1mm}

	\figcaption{Prenosová funkcia otvoreného regulačného obvodu v lineárnom uzavretom regulačnom obvode.}
    \label{schLinURO_basic_soro.pdf}
    }

\end{center}

\noindent
a~teda
\begin{align}
    G_{URO}(s) = \frac{G_{ORO}(s)}{1+G_{ORO}(s)} = \frac{G_R(s) G_S(s)}{1 + G_R(s) G_S(s)}
\end{align}
Tejto prenosovej funkcii sa tiež hovorí prenosová funkcia riadenia.


\subsection{Iné prenosové funkcie v URO}

Obdobne je možné skúmať aj iné pomery L-obrazov signálov v uzavretom regulačnom obvode. Napríklad tzv. prenosová funkcia regulačnej odchýlky
\begin{align}
    \frac{E(s)}{W(s)} = G_E(s)
\end{align}
V tomto prípade teda:

\begin{center}

    \vbox{%
    \makebox[\textwidth][c]{%
	\input{../SVG/schLinURO_GE.pdf_tex}
	}

	\figcaption{Schéma URO pre vyjadrenie prenosovej funkcie regulačnej odchýlky.}
    \label{schLinURO_GE.pdf}
    }

\end{center}

\noindent
a
\begin{align}
    G_E(s) = \frac{1}{1+G_{ORO}(s)} = \frac{1}{1 + G_R(s) G_S(s)}
\end{align}

Túto skutočnosť je možné využiť pri skúmaní dynamiky a~ustáleného stavu regulačnej odchýlky. Regulačná odchýlka totiž priamo hovorí o~splnení či nesplnení cieľa riadenia.

Typickým je tiež uvažovať tzv. poruchu akčného zásahu a~skúmať jej vplyv na výstup URO. Situácia vyzerá nasledovne:

\begin{center}

    \vbox{%
    \makebox[\textwidth][c]{%
	\input{../SVG/schLinURO_basic_porucha.pdf_tex}
	}

	\figcaption{Lineárny uzavretý regulačný obvod s uvažovaním poruchy akčného zásahu.}
    \label{schLinURO_basic_porucha.pdf}
    }

\end{center}

Pre izolovanie vplyvu poruchy na výstupnú veličinu je v prvom rade potrebné neuvažovať vplyv samotnej žiadanej hodnoty, teda $W(s) = 0$. Potom hovoríme a~prenosovej funkcii poruchy ak
\begin{align}
    \frac{Y(s)}{D(s)} = G_D(s)
\end{align}
teda:

\begin{center}

    \vbox{%
    \makebox[\textwidth][c]{%
	\input{../SVG/schLinURO_GD.pdf_tex}
	}

	\figcaption{Schéma URO pre vyjadrenie prenosovej funkcie poruchy.}
    \label{schLinURO_GD.pdf}
    }

\end{center}

\noindent
a
\begin{align}
    G_D(s) = \frac{G_S(s)}{1 + G_S(s) G_R(s)}
\end{align}










% -----------------------------------------------------------------------------

\end{document}

% -----------------------------------------------------------------------------