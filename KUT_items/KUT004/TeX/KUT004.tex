\documentclass[a4paper, 10pt, ]{article}

\usepackage[slovak]{babel}

% ------------------------------

\usepackage[utf8]{inputenc}
\usepackage[T1]{fontenc}

\usepackage[left=4cm,
            right=4cm,
            top=2.1cm,
            bottom=2.6cm,
            footskip=7.5mm,
            twoside,
            marginparwidth=3.0cm,
            %showframe,
            ]{geometry}

\usepackage{graphicx}
\usepackage[dvipsnames]{xcolor}
% https://en.wikibooks.org/wiki/LaTeX/Colors

% ------------------------------

\usepackage{lmodern}

\usepackage[tt={oldstyle=false,proportional=true,monowidth}]{cfr-lm}
% https://mirror.szerverem.hu/ctan/fonts/cfr-lm/doc/cfr-lm.pdf

% ------------------------------

\usepackage{amsmath}
\usepackage{amssymb}
\usepackage{amsthm}

\usepackage{booktabs}
\usepackage{multirow}
\usepackage{array}
\usepackage{dcolumn}

\usepackage{natbib}

% ------------------------------

\hyphenpenalty=6000
\tolerance=1000

\def\naT{\mathsf{T}}

% ------------------------------

\makeatletter

    \def\@seccntformat#1{\protect\makebox[0pt][r]{\csname the#1\endcsname\hspace{4mm}}}

    \def\cleardoublepage{\clearpage\if@twoside \ifodd\c@page\else
    \hbox{}
    \vspace*{\fill}
    \begin{center}
    \phantom{}
    \end{center}
    \vspace{\fill}
    \thispagestyle{empty}
    \newpage
    \if@twocolumn\hbox{}\newpage\fi\fi\fi}

    \newcommand\figcaption{\def\@captype{figure}\caption}
    \newcommand\tabcaption{\def\@captype{table}\caption}

\makeatother

% ------------------------------

\usepackage{fancyhdr}
\fancypagestyle{plain}{%
\fancyhf{} % clear all header and footer fields
% \fancyfoot[C]{\sffamily {\bfseries \thepage}\ | {\scriptsize\oznacenieCasti}}
\fancyfoot[C]{\sffamily {\bfseries \thepage}{\color{Gray}\scriptsize$\,$z$\,$\pageref{LastPage}}\ | \includegraphics[height=5pt]{./COMMONFILES/KUT_logo_v0.1.pdf}{\scriptsize\KUTporadoveCislo}}
\renewcommand{\headrulewidth}{0pt}
\renewcommand{\footrulewidth}{0pt}}
\pagestyle{plain}

% ------------------------------

\usepackage{titlesec}
\titleformat{\paragraph}[hang]{\sffamily  \bfseries}{}{0pt}{}
\titlespacing*{\paragraph}{0mm}{3mm}{1mm}
\titlespacing*{\subparagraph}{0mm}{3mm}{1mm}

\titleformat*{\section}{\sffamily\Large\bfseries}
\titleformat*{\subsection}{\sffamily\large\bfseries}
\titleformat*{\subsubsection}{\sffamily\normalsize\bfseries}


% ------------------------------

\PassOptionsToPackage{hyphens}{url}
\usepackage[pdfauthor={},
            pdftitle={},
            pdfsubject={},
            pdfkeywords={},
            % hidelinks,
            colorlinks=false,
            breaklinks,
            ]{hyperref}


% ------------------------------

\graphicspath{%
{../fig_standalone/}%
{../../PY/fig/}%
{../../ML/fig/}%
{./fig/}%
}

% ------------------------------

\usepackage{enumitem}

\usepackage{lettrine}

% ------------------------------

\usepackage{lastpage}

\usepackage{microtype}

% ------------------------------

\usepackage{algorithm}
\usepackage[noend]{algpseudocode}
\makeatletter
\renewcommand{\ALG@name}{Algoritmus}
\makeatother
\usepackage{amsmath}
\usepackage{bbold}
\usepackage{calc}
\usepackage{dsfont}
\usepackage{mathtools}
\usepackage{tabto}


\newcommand{\mr}[1]{\mathrm{#1}}
\newcommand{\bs}[1]{\boldsymbol{#1}}
\newcommand{\bm}[1]{\mathbf{#1}}

\newcommand{\diff}[2]{\frac{\Delta #1}{\Delta #2}}
\newcommand{\der}[2]{\frac{d #1}{d #2}}
\newcommand{\parder}[2]{\frac{\partial #1}{\partial #2}}

\newcommand{\argmax}[0]{\mr{argmax}}
\newcommand{\diag}[0]{\mr{diag}}
\newcommand{\rank}[0]{\mr{rank}}
\newcommand{\trace}[0]{\mr{tr}}

\renewcommand{\Re}{\mr{Re}}
\renewcommand{\Im}{\mr{Im}}


\theoremstyle{definition}
\newtheorem{definition}{Definícia}[section]
\newtheorem{theorem}{Veta}[section]
\newtheorem{lemma}[theorem]{Lemma}
\newtheorem{example}{Príklad}[section]
\renewcommand*{\proofname}{Dôkaz}

% ------------------------------


% -----------------------------------------------------------------------------

\def\oznacenieCelku{Kolekcia učebných textov}

% -----------------------------------------------------------------------------


\def\KUTporadoveCislo{004}

% \def\oznacenieVerzie{v0.9}
\def\oznacenieVerzie{\phantom{v1.0}}

\def\mesiacRok{apríl 2024}

\def\authorslabel{MT}





% -----------------------------------------------------------------------------

\begin{document}

% -----------------------------------------------------------------------------
% Uvodny nadpis

\noindent
\parbox[t][18mm][c]{0.3\textwidth}{%
\raisebox{-0.9\height}{%
\phantom{.}\includegraphics[height=18mm]{./COMMONFILES/URKFEIlogo.pdf}%
}%
}%
\parbox[t][18mm][c]{0.7\textwidth}{%
\raggedleft

\sffamily
\fontsize{16pt}{18pt}
\fontseries{sbc}
\selectfont

\noindent
\textcolor[rgb]{0.75, 0.75, 0.75}{\textls[25]{\oznacenieCelku}}
}%

\noindent
\parbox[t][16mm][b]{0.5\textwidth}{%
\raggedright

\color{Gray}
\sffamily

\fontsize{12pt}{12pt}
\selectfont
\mesiacRok

\fontsize{6pt}{10pt}
\selectfont
github.com/PracovnyBod/KUT

\fontsize{8pt}{10pt}
\selectfont
\authorslabel




}%
\parbox[t][16mm][b]{0.5\textwidth}{%
\raggedleft

\sffamily

\fontsize{6pt}{6pt}
\selectfont

\textcolor[rgb]{0.68, 0.68, 0.68}{\oznacenieVerzie}


\fontsize{14pt}{14pt}
\selectfont

\bfseries

\includegraphics[height=12pt]{./COMMONFILES/KUT_logo_v0.1.pdf}%
{%
\textls[-50]{\KUTporadoveCislo}
}%
}%

% -----------------------------------------------------------------------------




\vspace{6mm}

% ---------------------------------------------
\sffamily
\bfseries
\fontsize{18pt}{21pt}
\selectfont

\begin{flushleft}
	O PID regulátore
\end{flushleft}

\bigskip

% -----------------------------------------------------------------------------
\normalsize
\normalfont
% -----------------------------------------------------------------------------












\noindent
\lettrine[lines=1, nindent=1pt, loversize=0.0]{M}{edzi} 
najviac rozšírené súčasti riadiacich systémov vo všeobecnosti patrí PID regulátor. Využíva regulačnú odchýlku a je ho možné opísať pomocou lineárneho dynamického systému a teda pomocou prenosovej funkcie.


Názov \emph{PID} regulátor vystihuje skutočnosť, že tento regulátor má tri principiálne zložky. Proporcionálnu, Integračnú a derivačnú. Ide pri tom o tri spôsoby ako sa tu využíva informácia o regulačnej odchýlke.

Regulátor v skutočnosti pracuje s tromi signálmi. Prvým je samotná regulačná odchýlka $e(t) = w(t) - y(t)$. Z regulačnej odchýlky sa získavajú ďalšie dva signály. Časový integrál regulačnej odchýlky a časová derivácia regulačnej odchýlky. Formálnejšie
\begin{align}
    e_i(t) = \int e(t) \text{d}t
\end{align}
je časový integrál regulačnej odchýlky a
\begin{align}
    e_d(t) = \frac{\text{d}e(t)}{\text{d}t}
\end{align}
je časová derivácia regulačnej odchýlky.

Každý z týchto signálov je násobený (zosilnený) nejakou nastaviteľnou konštantou (parametrom regulátora) a výsledný akčný zásah $u(t)$ je súčet týchto troch členov, teda
\begin{align}
    u(t) = P e(t) + I  \int e(t) \text{d}t + D  \frac{\text{d}e(t)}{\text{d}t}
\end{align}
kde $P$, $I$ a $D$ sú parametre (konštanty, čísla, zosilnenia) regulátora.





\paragraph{Prenosová funkcia PID regulátora}


Zo všeobecného hľadiska je vstupom regulátora regulačná odchýlka -- signál $e(t)$. Jeho L-obrazom je $E(s)$. Výstupom regulátora je akčný zásah, ktorého L-obrazom je $U(s)$. Prenosová funkcia PID regulátora potom formálne je
\begin{align}
    G_R(s) = \frac{U(s)}{E(s)}
\end{align}
alebo teda akčný zásah je
\begin{align}
    U(s) = G_R(s) E(s)
\end{align}

L-obrazom integrálu regulačnej odchýlky je $\frac{1}{s} E(s)$ a L-obrazom derivácie regulačnej odchýlky je $sE(s)$. Potom akčný zásah je
\begin{align}
    U(s) = P E(s) + I  \frac{1}{s} E(s) + D  sE(s)
\end{align}
konvenčne sa v~tejto súvislosti pre označenie parametrov PID používajú $r_0$, $r_{-1}$ a~$r_1$, kde číselný index má vyjadrovať mocninu premennej $s$, pri ktorej sa parameter nachádza. Teda
\begin{align}
    U(s) = r_0 E(s) + r_{-1}  \frac{1}{s} E(s) + r_1  sE(s)
\end{align}

Prenosová funkcia PID regulátora je potom v tvare
\begin{align} \label{tfPID}
    G_R(s) = r_0  + r_{-1}  \frac{1}{s}  + r_1  s
\end{align}
Tento tvar sa nazýva zložkový tvar, totiž, túto prenosovú funkciu je možné vyjadriť aj v tvare
\begin{align}
    G_R(s) = P \left( 1 +  \frac{1}{T_I s}  +  T_D  s \right)
\end{align}
kde zmyslom je zavedenie časových konštánt $T_I$ a $T_D$ (tieto parametre majú rozmer času) a pri tom platí $P = r_0$, $T_I = \frac{r_0}{r_{-1}}$ a $T_I = \frac{r_1}{r_0}$.

Len pre názornosť, ak by sme chceli vyjadriť prenosovú funkciu \eqref{tfPID} ako jediný zlomok, potom
\begin{align} \label{tfPID2}
    G_R(s) =  \frac{ r_1 s^2 + r_0 s + r_{-1} }{s}
\end{align}
kde je zrejmé, že stupeň čitateľa je vyšší ako stupeň menovateľa a teda ide o~nekauzálny systém. To súvisí so skutočnosťou, že realizovať časovú deriváciu, takú, pri ktorej by bola veľkosť časového úseku $\text{d}t$ nekonečne malá, je nereálne. V praxi je možné realizovať kvalitnú aproximáciu ideálnej časovej derivácie, čo sa v týchto súvislostiach prejaví tak, že obrazom signálu v derivačnej zložke nie je $sE(s)$, ale iný výraz, taký, že to má za následok splnenie podmienky kauzality v celkovej prenosovej funkcii PID regulátora.






\paragraph{Bloková schéma PID regulátora}

Schematicky, pomocou základných funkčných prvkov, vzhľadom na jeho prenosovú funkciu, je možné PID regulátor znázorniť nasledovne:

\begin{center}

    \vspace{-1mm}

    \vbox{%
    \makebox[\textwidth][c]{%
    \input{../SVG/PID_blokSch.pdf_tex}
	}

    \vspace{-3mm}

	\figcaption{Bloková schéma PID regulátora}
    \label{PID_blokSch.pdf}
    }
	

    \vspace{-1mm}

\end{center}









% -----------------------------------------------------------------------------

\end{document}

% -----------------------------------------------------------------------------